\documentclass[]{article}

%opening
\title{Parental risk aversion and investment in children}
\author{Chau Pham}
%\date{} for no date shown
\usepackage{amsmath, amsfonts, amssymb, amsthm, mathtools, mathrsfs}
\usepackage[margin = 1.0in]{geometry}
\usepackage{bbold, soul}
\usepackage{tgpagella}
\usepackage{braket, setspace, parskip, enumitem}
\usepackage[colorlinks, citecolor = blue]{hyperref}
\usepackage[nameinlink, noabbrev]{cleveref}	% Use \Cref always 
\usepackage{caption}
\usepackage{threeparttable, makecell, cellspace, multirow, booktabs}
\usepackage{verbatim}
\usepackage{natbib}
\usepackage[bottom]{footmisc}	% keep footnotes at end of the page

%---- Simple table template
\begin{comment} 
	\begin{table}[!htbp]\centering
		\begin{threeparttable}
			\setlength{\extrarowheight}{0.2em}
			\begin{tabular}
				content...
			\end{tabular}
			%\caption{text}
			%\label{key}
		\end{threeparttable}
	\end{table}
\end{comment}

%\setlist[enumerate]{label=(\roman*)}

%\setlength{\parindent}{0pt}

%\captionsetup[figure]{labelfont=bf}
\captionsetup[table]{labelfont=bf}

\begin{document}

\maketitle
\onehalfspacing

\section{Data \& methodology}
I use data of National Longitudinal Survey of Youth 1979 (NLSY79) and NLSY79 Child and Young Adults (CYA) to study the effects of parental risk aversion on children investment. The NLSY79 is a longitudinal project that follows the lives of a sample of American youth born between 1957 and 1964 with the first round of data collection in 1979. The original cohort included 12,686 respondents ages 14-22 when first interviewed but later dropped to 9,964 respondents after the exclusion of the military subsample. Data are collected annually from 1979 to 1994 and biannually thereafter. Beginning in 1986, additional information was collected biannually about children born to female NLSY79 respondents, constituting the NLSY79 CYA dataset. The Child and Young Adult contains information on a rich set of assessments on cognitive ability, temperament, motor and social development, behaviour problems, self-competence of the children as well as the quality of their home environment. 

\subsection{Parental risk aversion}
To measure attitudes towards risk of parents,\footnote{More precisely, risk attitudes of mothers as NLSY79 and NLSY79 CYA are linked by identifying unique pairs of mother-child.} participants of NLSY79 were asked a series of hypothetical questions on occupational choice. The first question of the series reads:

\begin{center}
\begin{minipage}[!h]{.9\linewidth}\small
	``...Suppose that you are the only income earner in the family, and you have a good job guaranteed to give you your current (family) income every year for life. You are given the opportunity to take a new and equally good job, with a 50-50 chance that it will double your (family) income and a 50-50 chance that it will cut your (family) income by a third. Would you take the new job?"
\end{minipage}
\end{center}
\begin{flushright}
	\textit{Questionnaire Public Report, National Longitudinal Survey of Youth 1979 (NLSY79), 1993}
\end{flushright}


\begin{table}
	{
\def\sym#1{\ifmmode^{#1}\else\(^{#1}\)\fi}
\begin{tabular}{l*{4}{c}}
\toprule
                    &\multicolumn{1}{c}{All}&\multicolumn{1}{c}{HS dropout}&\multicolumn{1}{c}{HS}&\multicolumn{1}{c}{College}\\
\midrule
Moderately risk averse&       1.192         &       3.788         &       0.184         &       0.877         \\
                    &     (1.909)         &     (4.204)         &     (2.433)         &     (4.740)         \\
\addlinespace
Strongly risk averse&      -0.582         &       10.44\sym{**} &       0.704         &      -3.736         \\
                    &     (2.030)         &     (4.089)         &     (2.736)         &     (4.688)         \\
\addlinespace
Very strongly risk averse&       0.790         &       7.002\sym{**} &      -0.514         &       1.770         \\
                    &     (1.630)         &     (3.235)         &     (2.058)         &     (3.980)         \\
\midrule
Observations        &        5659         &        1046         &        2821         &        1300         \\
\bottomrule
\multicolumn{5}{l}{\footnotesize Standard errors in parentheses}\\
\multicolumn{5}{l}{\footnotesize \sym{*} \(p<0.10\), \sym{**} \(p<0.05\), \sym{***} \(p<0.01\)}\\
\end{tabular}
}
{
\def\sym#1{\ifmmode^{#1}\else\(^{#1}\)\fi}
\begin{tabular}{l*{4}{c}}
\toprule
                    &\multicolumn{1}{c}{All}&\multicolumn{1}{c}{HS dropout}&\multicolumn{1}{c}{HS}&\multicolumn{1}{c}{College}\\
\midrule
Moderately risk averse&      -0.298         &      -0.805         &       1.795         &      -1.416         \\
                    &     (1.810)         &     (4.098)         &     (2.293)         &     (3.770)         \\
\addlinespace
Strongly risk averse&      -0.523         &       0.925         &      -0.270         &      -0.820         \\
                    &     (1.817)         &     (4.003)         &     (2.317)         &     (3.812)         \\
\addlinespace
Very strongly risk averse&      -1.211         &       3.578         &      0.0140         &      -4.169         \\
                    &     (1.470)         &     (2.984)         &     (1.844)         &     (3.102)         \\
\midrule
Observations        &        5877         &        1084         &        3104         &        1401         \\
\bottomrule
\multicolumn{5}{l}{\footnotesize Standard errors in parentheses}\\
\multicolumn{5}{l}{\footnotesize \sym{*} \(p<0.10\), \sym{**} \(p<0.05\), \sym{***} \(p<0.01\)}\\
\end{tabular}
}
{
\def\sym#1{\ifmmode^{#1}\else\(^{#1}\)\fi}
\begin{tabular}{l*{4}{c}}
\toprule
                    &\multicolumn{1}{c}{All}&\multicolumn{1}{c}{HS dropout}&\multicolumn{1}{c}{HS}&\multicolumn{1}{c}{College}\\
\midrule
Moderately risk averse&      0.0130         &     0.00913         &      0.0252         &    -0.00806         \\
                    &    (0.0474)         &     (0.107)         &    (0.0594)         &     (0.101)         \\
\addlinespace
Strongly risk averse&      0.0153         &      0.0960         &      0.0316         &     -0.0984         \\
                    &    (0.0476)         &     (0.109)         &    (0.0603)         &     (0.101)         \\
\addlinespace
Very strongly risk averse&     -0.0227         &      0.0979         &    -0.00585         &      -0.141         \\
                    &    (0.0403)         &    (0.0947)         &    (0.0496)         &    (0.0883)         \\
\midrule
Observations        &        5772         &        1099         &        3282         &        1457         \\
\bottomrule
\multicolumn{5}{l}{\footnotesize Standard errors in parentheses}\\
\multicolumn{5}{l}{\footnotesize \sym{*} \(p<0.10\), \sym{**} \(p<0.05\), \sym{***} \(p<0.01\)}\\
\end{tabular}
}

\end{table}

\pagebreak

%\bibliographystyle{apalike}
%\bibliography{bibliography.bib}

\end{document}
