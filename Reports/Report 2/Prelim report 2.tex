\documentclass[]{article}

%opening
\title{Parental risk aversion and investment in children}
\author{Chau Pham}
%\date{} for no date shown
\usepackage{amsmath, amsfonts, amssymb, amsthm, mathtools, mathrsfs}
\usepackage[margin = 1.0in]{geometry}
\usepackage{bbold, soul}
\usepackage{tgpagella}
\usepackage{braket, setspace, parskip, enumitem}
\usepackage[colorlinks, citecolor = blue]{hyperref}
\usepackage[nameinlink, noabbrev]{cleveref}	% Use \Cref always 
\usepackage{caption}
\usepackage{threeparttable, makecell, cellspace, multirow, booktabs, diagbox, tabularx, longtable, threeparttablex}
\usepackage{verbatim}
\usepackage{natbib}
\usepackage[bottom]{footmisc}	% keep footnotes at end of the page

\newcolumntype{Y}{>{\centering\arraybackslash}X}

\renewcommand\theadalign{bc}
\renewcommand\theadfont{\bfseries}
\renewcommand\theadgape{\Gape[4pt]}
\renewcommand\cellgape{\Gape[4pt]}

%---- Simple table template
\begin{comment} 
	\begin{table}[!htbp]\centering
		\begin{threeparttable}
			\setlength{\extrarowheight}{0.2em}
			\begin{tabular}
				content...
			\end{tabular}
			%\caption{text}
			%\label{key}
		\end{threeparttable}
	\end{table}
\end{comment}

%\setlist[enumerate]{label=(\roman*)}

%\setlength{\parindent}{0pt}

%\captionsetup[figure]{labelfont=bf}
\captionsetup[table]{labelfont=bf}

\begin{document}

\maketitle
\onehalfspacing

\section{Data \& methodology}
I use data of National Longitudinal Survey of Youth 1979 (NLSY79) and NLSY79 Child and Young Adults (CYA) to study the effects of parental risk aversion on children investment. The NLSY79 is a longitudinal project that follows the lives of a sample of American youth born between 1957 and 1964 with the first round of data collection in 1979. The original cohort included 12,686 respondents ages 14-22 when first interviewed but later dropped to 9,964 respondents after the exclusion of the military subsample. Data are collected annually from 1979 to 1994 and biannually thereafter. Beginning in 1986, additional information was collected biannually about children born to female NLSY79 respondents, constituting the NLSY79 CYA dataset. The Child and Young Adult contains information on a rich set of assessments on cognitive ability, temperament, motor and social development, behaviour problems, self-competence of the children as well as the quality of their home environment. By linking investment, as measured by The HOME (Home Observation Measurement of the Environment) indices along with its subsection indices in the NLSY79 CYA, and an indicator of risk attitudes generated from a series of questions in the NLSY79, it is possible to study the effect of parental (maternal) risk aversion on investment in children.

\subsection{Parental risk aversion}
To measure attitudes towards risk of parents,\footnote{More precisely, risk attitudes of mothers as NLSY79 and NLSY79 CYA are linked by identifying unique pairs of mother-child.} participants of NLSY79 were asked a series of hypothetical questions on occupational choice. The first question of the series reads:

\begin{center}
\begin{minipage}[!h]{.9\linewidth}\small
	``...Suppose that you are the only income earner in the family, and you have a good job guaranteed to give you your current (family) income every year for life. You are given the opportunity to take a new and equally good job, with a 50-50 chance that it will double your (family) income and a 50-50 chance that it will cut your (family) income by a third. Would you take the new job?"
\end{minipage}
\end{center}
\begin{flushright}
	\textit{Questionnaire Public Report, National Longitudinal Survey of Youth 1979 (NLSY79), 1993.}
\end{flushright}
Depending on respondents' answers to this question, survey participants were asked a similar question holding the first choice constant while the income cut in the second choice is modified. If participants choose ``First job" in the first question, the income cut in the second question is lowered to 20\%. If ``Second job" is chosen in the first question, the income cut is increased to 1/2 in the second question. The first series of questions on risk attitudes were asked in four rounds in 1993, 2002, 2004 and 2006. In 2010, 2012, and 2014, roughly 1,800 survey participants who had not been interviewed in previous rounds were asked a slightly modified version of the original questions.\footnote{The majority of these participants responded to the survey in 2010. Only those who did not were asked in following rounds. After various restrictions on data availability for investment variables and other controls, roughly 900 observations all collected in 2010 are left for the new series.} The first question of the ``new" series reads:

\begin{center}
\begin{minipage}{.9\linewidth}\small
	``Suppose that you are the only income earner in the family, and that you have to choose between two new jobs. The first job would guarantee your current total family income for life. The second job is possibly better paying, but the income is also less certain. There is a 50-50 chance the second job would increase your total lifetime income by 20 percent and a 50-50 change that is would cut it by 10 percent."
\end{minipage}
\end{center} 
\begin{flushright}
	\textit{Questionnaire Public Report, National Longitudinal Survey of Youth 1979 (NLSY79), 2010.}
\end{flushright}
Choosing ``First job" would lead respondents to a second question with income cut increased to 15\% as opposed to 5\% for ``Second job". Unlike the previous series, however, respondents to the ``new" series are only interviewed once. 

Using these two series on risk attitudes, a 4-level categorical variable denoting degrees of risk aversion is created. The specifics of the created variable is shown in \Cref{table:1}. As all the hypothetical lotteries on job returns deliver positive gains in expectation, it is only possible to classify those who choose ``Second job" in both questions they are presented as ``Weakly risk averse." Since there are no questions with the derived hypothetical lottery that pays 0 in expectation or involves expected loss, little can be inferred about those who are risk neutral or risk loving. 

\begin{table}[!h]
	\centering
	\setlength{\extrarowheight}{0.2em}
	\caption{Indicator of risk attitudes}	
	{
\begin{threeparttable}
	\begin{tabular}{|l| *{3}{c|}}
		\hline
		\diagbox{Risk aversion}{Question} & Original & Increased income cut\tnote{a} & Reduced income cut\tnote{b} \\ \hline
		Very strongly risk averse & First job & - & First job\\
		Strongly risk averse & First job & - & Second job\\
		Moderately risk averse & Second job & First job & -\\
		Weakly risk averse & Second job & Second job & - \\
		\hline
	\end{tabular}
	\begin{tablenotes}\footnotesize
		\item[a] This includes both series. Income cut increased to 50\% in the old series and 15\% in the new series.
		\item[b] This includes both series. Income cut decreased to 20\% in the old series and 5\% in the new series.
	\end{tablenotes}
\end{threeparttable}

}
	\label{table:1}
\end{table}



\subsection{Investment data}
Investment in children can be captured by parental inputs measured by HOME indices. HOME (Short Form) comprises of primary measures of the quality of a child's home environment included in the NLSY79 Child survey. Various components of the HOME evaluate the cognitive stimulation and emotional support children below 15 years old receive from the home environment. Items that made up the HOME score are grouped into goods input and time input. These are reported in \Cref{table:2}. Apart from measures of HOME section, several items included in the self-administered questions for children between 10-14 are also present and have the (NH) prefix.  

For each age group, to generate the generic investment index, indicators of all items in \Cref{table:2} for the corresponding age are totalled then normalised to have a mean zero and one standard deviation. A similar procedure is also applied to goods investment index and time investment index. Other assessments of cognitive stimulation and emotional support constitute cognitive stimulation index and emotional support index. Both of which are measured in percentages.

\begin{table}
	\centering
	\setlength{\extrarowheight}{0.2em}
	\caption{Components of HOME index}	
	{
\begin{threeparttable}
	\begin{tabularx}{\textwidth}{|l| *{4}{Y|}}
		\hline 
		\diagbox[width = .67\linewidth]{Items}{Age group} & 0-2 & 3-5 & 6-9 & 10-14 \\
		\hline
		\textit{Goods} & & & & \\
		\hspace{3mm} Child has 10 or more soft toys at home & x & & & \\
		\hspace{3mm} Child has 10 or more push/pull toys at home & x & & & \\
		\hspace{3mm} Child has more than 10 books at home & x & x & x & x \\
		\hspace{3mm} Family gets at least three magazines regularly & & x & & \\
		\hspace{3mm} Child has a  CD player & & x & & \\
		\hspace{3mm} Family subscribes to daily newspaper & & & x & x \\
		\hspace{3mm} Child has a musical instrument & & & x & x \\
		& & & & \\
		\textit{Time} & & & & \\
		\hspace{3mm} Child taken to grocery at least once a week & x & & & \\
		\hspace{3mm} Child goes on outings more than three times per month & x & x & & \\
		\hspace{3mm} Child east at least one meal per day with both parents & x & x & x & x \\
		\hspace{3mm} Child sees father(-figure) daily & x & x & x & x \\
		\hspace{3mm} Mother reads to child at least once a week & x & x & x & \\
		\hspace{3mm} Child goes to museum more than twice in past year & & x & x & x \\
		\hspace{3mm} Child spends time with father(-figure) at least four times a week &  &  & x & x \\
		\hspace{3mm} Family gets together with friends/relatives at least twice a month & & & x & x \\
		\hspace{3mm} Child spends time with father(-figure) outdoor once a week & & & x & x \\
		\hspace{3mm} Mother discusses TV programmes with child & & & x & x \\
		\hspace{3mm} Child goes to theatre/performance more than twice in past year & & & x & x \\
		& & & & \\
		\textit{Activities last month\tnote{a}:} & & & & \\
		\hspace{3mm} (NH) Child went shopping with parents & & & & x \\
		\hspace{3mm} (NH) Child went on an outing with parents & & & & x \\
		\hspace{3mm} (NH) Child went with parents to movies & & & & x \\
		\hspace{3mm} (NH) Child went with parents to dinner& & & & x \\
		& & & & \\
		\textit{Activities last week\tnote{a}:} & & & & \\
		\hspace{3mm} (NH) Child worked with parents on schoolwork & & & & x  \\
		\hspace{3mm} (NH) Child did things together with parents & & & & x  \\	
		\hspace{3mm} (NH) Child play game or sports together with parents & & & & x  \\
		\hline			
	\end{tabularx}
	\begin{tablenotes}\footnotesize
		\item[a] These items are not from HOME but from self-administered survey of children
	\end{tablenotes}
\end{threeparttable}
}
	\label{table:2}
\end{table}  

\subsection{Summary statistics}
Descriptive statistics of variables used in our analysis are reported in \Cref{table:3}. 

\begin{ThreePartTable}
	\centering
	\setlength{\extrarowheight}{0.2em}
	\begin{TableNotes}\footnotesize
		\item[a] Mean and standard deviations enclosed in parentheses are reported for all numerical variables. Only frequency is reported for categorical variables.
	\end{TableNotes}
	\begin{table}[htbp]\centering \caption{Summary statistics \label{table:5-summary}} \begin{tabular}{l*{1}{cc}} \toprule
                    &\multicolumn{2}{c}{}     \\
\midrule
IMPUTED\_CRRA\_1      &        1.42&      (0.91)\\
IMPUTED\_CRRA\_4      &        1.15&      (0.77)\\
HOME index          &        0.11&      (0.96)\\
Goods index         &        0.08&      (0.98)\\
Time index          &        0.08&      (0.97)\\
Cognitive Stimulation score (\%)&       48.88&     (29.11)\\
Emotional Support score (\%)&       49.95&     (29.23)\\
Number of siblings  &        1.61&      (1.23)\\
Family size         &        4.30&      (1.39)\\
$\#$ weeks worked last year&       32.17&     (22.59)\\
$\#$ weeks spouse worked last year&       48.55&      (9.75)\\
Net household income (\$)&    50883.68&  (78376.30)\\
\bottomrule
\end{tabular}
\end{table}
\begin{tabular}{l*{1}{cc}} \hline\hline
                    &\multicolumn{2}{c}{}     \\
\hline
Female              &        0.49&      (0.50)\\
Black               &        0.25&      (0.43)\\
Non-Black, non-Hispanic&        0.54&      (0.50)\\
Age of mother at child's birth&       27.61&      (5.60)\\
Age of mother at first birth&       23.67&      (5.39)\\
Mother's AFQT score (2006)&       42.95&     (28.52)\\
Mother lives with both biological parents at age 14&        0.74&      (0.44)\\
Mother's mother's years of schooling&       10.73&      (3.28)\\
Mother's father's years of schooling&       10.63&      (4.10)\\
\hline\hline
\end{tabular}

\end{ThreePartTable}

\section{Preliminary results}
\subsection{Simple regression}
OLS estimates of the relationship between parental risk aversion and investment in children are reported in \Cref{table:4}. The particular regression equation takes the form:
\begin{equation}
	Y_{it} = c + \alpha RISK_{it} + \beta Z_{it} + \gamma_t +  \varepsilon_{it}
\end{equation}
where $Y_{it}$ are measurements of parental inputs, namely HOME index and its subcomponents. $Z_{it}$ is a vector of controls for child $i$ at year $t$. This includes both child's characteristics and mother's characteristics as reported in \Cref{table:3}. I also augment the equation by adding year and age category fixed effects.\footnote{Since HOME index along with its subcomponents are calculated and normalised by age category (see \Cref{table:2}) rather than age, I decided to include age-category as a fixed effect for the model.} $RISK_{it}$ is a 4-level categorical variable derived from the question on occupational choice for mother of child $i$ at time $t$. $\gamma_t$ denotes time fixed effect. Standard errors are clustered by mother. 

Results indicate heterogeneous effects of risk aversion on parental inputs. The effects are heterogeneous both in the dimensions of investment as well as the signs of impact at varying degree of risk aversion. At a glance, children of those who never attended high school tend to receive less investment the more risk averse their mothers are. On the other hand, the opposite trend is found for children of high school dropouts, children of more risk averse mothers tend to receive more investment. Results for the other two subsamples of mothers with high school degree and above are weak and mostly insignificant. 

\begin{table}
	\centering \setlength{\extrarowheight}{0.1em}
	\caption{OLS Estimates of the effect of Risk Aversion on Parental Inputs}
	{
\def\sym#1{\ifmmode^{#1}\else\(^{#1}\)\fi}
\begin{tabular}{l*{5}{c}}
\toprule
                    &\multicolumn{1}{c}{\thead{All}}&\multicolumn{1}{c}{\thead{No HS}}&\multicolumn{1}{c}{\thead{HS dropout}}&\multicolumn{1}{c}{\thead{HS}}&\multicolumn{1}{c}{\thead{College \\and above}}\\
\midrule
\textit{Panel A: HOME index} & & & & & \\
Moderately risk averse&      0.0476         &      -0.328\sym{*}  &       0.246         &      0.0929         &      0.0268         \\
                    &    (0.0513)         &     (0.166)         &     (0.163)         &    (0.0681)         &    (0.0909)         \\
Strongly risk averse&      0.0399         &      -0.225         &       0.317\sym{*}  &      0.0855         &     -0.0236         \\
                    &    (0.0493)         &     (0.149)         &     (0.171)         &    (0.0662)         &    (0.0866)         \\
Very strongly risk averse&     0.00632         &      -0.329\sym{**} &       0.318\sym{**} &      0.0135         &     -0.0238         \\
                    &    (0.0445)         &     (0.138)         &     (0.148)         &    (0.0575)         &    (0.0809)         \\
\midrule
Observations        &        5806         &         399         &         585         &        2931         &        1891         \\
\midrule
\textit{Panel B: Goods index} & & & & & \\
Moderately risk averse&      0.0329         &     -0.0996         &     -0.0304         &      0.0349         &      0.0956         \\
                    &    (0.0526)         &     (0.161)         &     (0.174)         &    (0.0747)         &    (0.0788)         \\
Strongly risk averse&      0.0607         &      -0.202         &       0.157         &      0.0343         &       0.149\sym{*}  \\
                    &    (0.0504)         &     (0.153)         &     (0.169)         &    (0.0761) & (0.0759)\\        
Very strongly risk averse&      0.0587         &      -0.173         &       0.136         &      0.0126         &       0.139\sym{**} \\
                    &    (0.0438)         &     (0.136)         &     (0.155)         &    (0.0606)         &    (0.0683)         \\
\midrule
Observations        &        5548         &         386         &         563         &        2798         &        1801         \\
\midrule
\textit{Panel C: Time index} & & & & & \\
Moderately risk averse&      0.0313         &      -0.444\sym{***}&       0.247         &       0.109         &     -0.0217         \\
                    &    (0.0514)         &     (0.149)         &     (0.159)         &    (0.0687)         &    (0.0913)         \\

Strongly risk averse&     0.00546         &      -0.201         &       0.242         &      0.0802         &      -0.107         \\
                    &    (0.0502)         &     (0.146)         &     (0.174)         &    (0.0646)         &    (0.0900)         \\

Very strongly risk averse&     -0.0139         &      -0.404\sym{***}&       0.316\sym{**} &      0.0341         &     -0.0920         \\
                    &    (0.0445)         &     (0.131)         &     (0.140)         &    (0.0583)         &    (0.0817)         \\
\midrule
Observations        &        5548         &         386         &         563         &        2798         &        1801         \\
\midrule
\textit{Panel D: Cognitive Stimulation score (\%)} & & & & & \\
Moderately risk averse&       0.929         &      -9.852\sym{**} &       5.223         &       3.441         &      -0.904         \\
                    &     (1.816)         &     (4.873)         &     (5.401)         &     (2.459)         &     (3.076)         \\

Strongly risk averse&       0.410         &      -6.944         &       9.780\sym{*}  &       0.587         &      -0.432         \\
                    &     (1.812)         &     (6.257)         &     (5.635)         &     (2.533)         &     (2.996)         \\

Very strongly risk averse&      -0.291         &      -8.434\sym{**} &       10.05\sym{**} &       0.372         &      -2.719         \\
                    &     (1.539)         &     (4.238)         &     (4.572)         &     (2.058)         &     (2.637)         \\
\midrule
Observations        &        5162         &         346         &         521         &        2627         &        1668         \\
\midrule
\textit{Panel E: Emotional Support score (\%)} & & & & & \\
Moderately risk averse&      -0.852         &      -15.68\sym{**} &       11.71\sym{**} &       0.923         &      -3.737         \\
                    &     (1.939)         &     (6.717)         &     (5.677)         &     (2.719)         &     (3.485)         \\

Strongly risk averse&      -2.277         &       2.429         &       11.50\sym{**} &      -1.097         &      -7.397\sym{**} \\
                    &     (1.962)         &     (5.214)         &     (5.685)         &     (2.818)         &     (3.499)         \\
Very strongly risk averse&      -0.745         &      -2.830         &       10.53\sym{**} &      -0.879         &      -3.206         \\
                    &     (1.629)         &     (5.271)         &     (4.780)         &     (2.182)         &     (3.085)         \\
\midrule
Observations        &        4713         &         306         &         475         &        2402         &        1530         \\
\bottomrule
\multicolumn{6}{l}{\footnotesize Standard errors in parentheses}\\
\multicolumn{6}{l}{\footnotesize \sym{*} \(p<0.10\), \sym{**} \(p<0.05\), \sym{***} \(p<0.01\)}\\
\end{tabular}
}

	\label{table:4}
\end{table}

\pagebreak 

\subsection{Separating old and new measures of risk aversion}
It is important to note that the two series involve different stakes of gains and losses. Consequently, degree of risk aversion may differ considerably between those answering old series and those answering new series even if they might belong to the same category of risk aversion. To account for this potential erroneous categorisation, I add an interaction term between the risk aversion variable and an indicator denoting old and new series of question. Results are reported in \Cref{table:5}. One caveat of this approach, however, is that since data for the new series are all collected in 2010, there is a high degree of collinearity for the interaction terms. Specifically, the interaction term between very strongly risk averse and the indicator is omitted. 

\pagebreak
    
\begin{ThreePartTable}
	\centering
	\setlength{\extrarowheight}{0.1em}
	\begin{TableNotes}\footnotesize
		\item ... 
	\end{TableNotes}
	{
\def\sym#1{\ifmmode^{#1}\else\(^{#1}\)\fi}
\begin{longtable}{l*{5}{c}}
\caption{OLS estimates with interaction term\label{table:5}}\\
\toprule\endfirsthead\midrule\endhead\midrule\endfoot\endlastfoot
                    &\multicolumn{1}{c}{\thead{All}}&\multicolumn{1}{c}{\thead{No HS}}&\multicolumn{1}{c}{\thead{HS dropout}}&\multicolumn{1}{c}{\thead{HS}}&\multicolumn{1}{c}{\thead{College\\ and above}}\\
\midrule
\textit{Panel A: HOME index} & & & & & \\
Moderately risk averse&      0.0174         &      -0.347\sym{**} &       0.211         &      0.0737         &     -0.0426         \\
                    &    (0.0516)         &     (0.169)         &     (0.171)         &    (0.0681)         &    (0.0918)         \\

Strongly risk averse&      0.0332         &      -0.389\sym{**} &       0.285         &      0.0932         &     -0.0424         \\
                    &    (0.0520)         &     (0.165)         &     (0.187)         &    (0.0702)         &    (0.0868)         \\

Very strongly risk averse&    -0.00504         &      -0.335\sym{**} &       0.281\sym{*}  &      0.0268         &     -0.0788         \\
                    &    (0.0458)         &     (0.139)         &     (0.165)         &    (0.0591)         &    (0.0811)         \\

Weakly risk averse $\times$ New&      -0.139         &      -0.231         &      -0.557         &       0.168         &      -0.487\sym{**} \\
                    &     (0.154)         &     (0.440)         &     (0.504)         &     (0.238)         &     (0.242)         \\

Moderately risk averse $\times$ New&       0.336\sym{*}  &      -0.310         &    -0.00340         &       0.511\sym{*}  &       0.432\sym{**} \\
                    &     (0.188)         &     (0.332)         &     (0.324)         &     (0.305)         &     (0.180)         \\

Strongly risk averse $\times$ New&     -0.0239         &       1.100\sym{***}&     -0.0469         &      0.0681         &      -0.268\sym{*}  \\
                    &    (0.0963)         &     (0.268)         &     (0.391)         &     (0.130)         &     (0.160)         \\
\midrule
Observations        &        5806         &         399         &         585         &        2931         &        1891         \\

\midrule
\textit{Panel B: Goods index} & & & & & \\
Moderately risk averse&     0.00911         &      -0.157         &      -0.105         &      0.0312         &      0.0624         \\
                    &    (0.0537)         &     (0.171)         &     (0.190)         &    (0.0778)         &    (0.0770)         \\

Strongly risk averse&      0.0597         &      -0.302\sym{*}  &      0.0809         &      0.0610         &       0.136\sym{*}  \\
                    &    (0.0532)         &     (0.173)         &     (0.194)         &    (0.0808)         &    (0.0757)         \\

Very strongly risk averse&      0.0565         &      -0.220         &      0.0954         &      0.0295         &       0.125\sym{*}  \\
                    &    (0.0448)         &     (0.148)         &     (0.178)         &    (0.0629)         &    (0.0647)         \\

Weakly risk averse $\times$ New&     -0.0366         &      -0.846\sym{**} &      -0.634\sym{**} &       0.256         &      -0.112         \\
                    &     (0.166)         &     (0.356)         &     (0.303)         &     (0.212)         &     (0.266)         \\

Moderately risk averse $\times$ New&       0.368\sym{***}&      0.0838         &       0.451         &       0.328         &       0.405\sym{**} \\
                    &     (0.124)         &     (0.278)         &     (0.288)         &     (0.203)         &     (0.159)         \\

Strongly risk averse $\times$ New&     0.00538         &       0.370         &       0.237         &     -0.0311         &     0.00160         \\
                    &    (0.0965)         &     (0.356)         &     (0.245)         &     (0.156)         &     (0.129)         \\
\midrule
Observations        &        5548         &         386         &         563         &        2798         &        1801         \\


\midrule
\textit{Panel C: Time index} & & & & & \\
Moderately risk averse&     0.00856         &      -0.395\sym{**} &       0.202         &      0.0895         &     -0.0779         \\
                    &    (0.0516)         &     (0.155)         &     (0.160)         &    (0.0682)         &    (0.0953)         \\

Strongly risk averse&     0.00510         &      -0.305\sym{*}  &       0.239         &      0.0766         &      -0.113         \\
                    &    (0.0524)         &     (0.159)         &     (0.175)         &    (0.0682)         &    (0.0959)         \\

Very strongly risk averse&     -0.0228         &      -0.335\sym{**} &       0.270\sym{*}  &      0.0427         &      -0.142         \\
                    &    (0.0456)         &     (0.138)         &     (0.150)         &    (0.0593)         &    (0.0863)         \\

Weakly risk averse $\times$ New&      -0.112         &       0.727\sym{**} &      -0.617         &      0.0990         &      -0.454\sym{**} \\
                    &     (0.158)         &     (0.342)         &     (0.518)         &     (0.259)         &     (0.213)         \\

Moderately risk averse $\times$ New&       0.254         &      -0.366         &      0.0209         &       0.432         &       0.261         \\
                    &     (0.197)         &     (0.365)         &     (0.461)         &     (0.330)         &     (0.187)         \\

Strongly risk averse $\times$ New&     -0.0505         &       1.153\sym{***}&      -0.288         &       0.104         &      -0.317\sym{**} \\
                    &    (0.0962)         &     (0.272)         &     (0.421)         &     (0.123)         &     (0.148)         \\
\midrule
Observations        &        5548         &         386         &         563         &        2798         &        1801         \\

\midrule
\textit{Panel D: Cognitive Stimulation score (\%)} & & & & & \\
Moderately risk averse&       0.453         &      -10.49\sym{**} &       5.301         &       2.985         &      -2.311         \\
                    &     (1.888)         &     (5.159)         &     (5.660)         &     (2.555)         &     (3.116)         \\

Strongly risk averse&       0.368         &      -11.27         &       10.53\sym{*}  &       0.181         &      -0.739         \\
                    &     (1.967)         &     (7.098)         &     (5.919)         &     (2.730)         &     (3.200)         \\

Very strongly risk averse&      -0.418         &      -9.368\sym{**} &       11.06\sym{**} &       0.586         &      -4.300         \\
                    &     (1.606)         &     (4.584)         &     (4.738)         &     (2.151)         &     (2.667)         \\

Weakly risk averse $\times$ New&      -1.711         &      -15.69\sym{**} &       20.18\sym{***}&       2.209         &      -14.97\sym{**} \\
                    &     (4.782)         &     (7.856)         &     (7.673)         &     (7.162)         &     (7.377)         \\

Moderately risk averse $\times$ New&       6.123         &      -19.08         &       9.383         &       9.919         &       1.163         \\
                    &     (4.702)         &     (15.23)         &     (10.51)         &     (7.206)         &     (6.624)         \\

Strongly risk averse $\times$ New&      -0.371         &       26.00\sym{**} &       2.835         &       4.603         &      -10.31\sym{**} \\
                    &     (3.758)         &     (10.98)         &     (11.24)         &     (6.199)         &     (4.790)         \\
\midrule
Observations        &        5162         &         346         &         521         &        2627         &        1668         \\


\midrule
\textit{Panel E: Emotional Support score (\%)} & & & & & \\
Moderately risk averse&      -0.233         &      -15.58\sym{**} &       11.18\sym{*}  &       1.529         &      -2.874         \\
                    &     (2.021)         &     (6.885)         &     (6.084)         &     (2.813)         &     (3.738)         \\

Strongly risk averse&      -1.464         &      -0.666         &       13.03\sym{**} &     -0.0670         &      -6.767\sym{*}  \\
                    &     (2.138)         &     (5.396)         &     (6.129)         &     (3.045)         &     (3.975)         \\

Very strongly risk averse&       0.157         &      -3.413         &       10.37\sym{**} &       0.433         &      -2.505         \\
                    &     (1.728)         &     (5.420)         &     (5.060)         &     (2.295)         &     (3.405)         \\

Weakly risk averse $\times$ New&       10.59\sym{**} &      -14.85         &      -1.074         &       18.01\sym{***}&       5.955         \\
                    &     (4.238)         &     (11.26)         &     (8.182)         &     (6.348)         &     (6.483)         \\

Moderately risk averse $\times$ New&       3.540         &      -18.00         &       3.728         &       9.601         &      -4.401         \\
                    &     (5.288)         &     (10.84)         &     (11.77)         &     (7.747)         &     (9.502)         \\

Strongly risk averse $\times$ New&       1.389         &       24.68\sym{**} &      -9.204         &       3.277         &       0.890         \\
                    &     (3.819)         &     (9.760)         &     (14.08)         &     (5.228)         &     (6.125)         \\
\midrule
Observations        &        4713         &         306         &         475         &        2402         &        1530         \\
\bottomrule
\multicolumn{6}{l}{\footnotesize Standard errors in parentheses}\\
\multicolumn{6}{l}{\footnotesize \sym{*} \(p<0.10\), \sym{**} \(p<0.05\), \sym{***} \(p<0.01\)}\\
\end{longtable}
}

\end{ThreePartTable}


\pagebreak

%\bibliographystyle{apalike}
%\bibliography{bibliography.bib}

\end{document}
