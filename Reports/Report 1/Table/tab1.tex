
% Table created by stargazer v.5.2.3 by Marek Hlavac, Social Policy Institute. E-mail: marek.hlavac at gmail.com
% Date and time: Mon, Jun 26, 2023 - 20:41:51
\begin{table}[!b] \centering 
  \caption{Summary Statistics (LISS)} 
  \label{tab:1} 
  \begin{threeparttable}

\begin{tabular}{@{\extracolsep{5pt}}lccccc} 
\\[-1.8ex]\hline 
\hline \\[-1.8ex] 
Statistic & \multicolumn{1}{c}{N} & \multicolumn{1}{c}{Mean} & \multicolumn{1}{c}{St. Dev.} & \multicolumn{1}{c}{Min} & \multicolumn{1}{c}{Max} \\ 
\hline \\[-1.8ex] 
\tnote{1}\hspace{3mm}Monthly school fees (euros) & 161 & 105.484 & 317.641 & 0 & 3,200 \\ 
\tnote{1}\hspace{3mm}Financial responsibility  & 161 & 0.522 & 0.501 & 0 & 1 \\ 
Have a PC & 161 & 0.037 & 0.190 & 0 & 1 \\ 
Household income & 148 & 2,986.588 & 1,156.707 & 500 & 7,000 \\ 
Number of living-at-home children & 161 & 1.925 & 0.763 & 1 & 5 \\
Risk aversion ($\rho$) & 161 & 2.513 & 8.948 & $-$4.836 & 38.579 \\ 
\tnote{2}\hspace{3mm}$AA_{0.1}$ & 161 & $-$0.101 & 0.219 & $-$0.872 & 0.075 \\ 
\tnote{2}\hspace{3mm}$AA_{0.5}$ & 161 & 0.091 & 0.218 & $-$0.469 & 0.469 \\ 
\tnote{2}\hspace{3mm}$AA_{0.9}$ & 161 & 0.606 & 0.304 & $-$0.072 & 0.875 \\ 
Index $a$ & 161 & $-$0.005 & 0.523 & $-$1.184 & 1.184 \\ 
Index $b$ & 161 & 0.617 & 0.292 & 0.096 & 1.942 \\ 
\tnote{3}\hspace{3mm}Education\\
\hspace{5mm} Primary& 4 & [2.45\%] && & \\
\hspace{5mm} Intermediate secondary school (vmbo) & 28 & [17.39\%] & & & \\
\hspace{5mm} Higher secondary school (havo/vwo) & 17 & [10.56\%]& & & \\
\hspace{5mm} Intermediate vocational education (mbo) & 50 & [31.06\%]& & & \\
\hspace{5mm} Higher vocational education (hbo) & 17 & [10.56\%]& & & \\
\hspace{5mm} University (wo) & 43 & [26.71\%]& & & \\
\hspace{5mm} Other & 19 & [11.8\%]& & & \\
\hline \\[-1.8ex] 
\end{tabular} 
\begin{tablenotes}
	\footnotesize
	\item[1] These are dummies indicating respectively "whether the respondent make financial decisions in the household" and "whether the respondent has a PC."  
	\item[2] $AA_p$ denotes the \textit{local ambiguity index} for the ambiguity-neutral probability $p$, calculated as $AA_p=p-m(p)$.
	\item[3] The figures in brackets are the proportions of respondents belonging to the corresponding education level.
\end{tablenotes}
  \end{threeparttable}
\end{table} 
