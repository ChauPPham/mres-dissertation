\documentclass[emulatestandardclasses, 10pt, abstract = true]{scrartcl}


%-------------------------- packages
\usepackage{amsmath, amsfonts, amssymb, amsthm, mathtools, mathrsfs}
\usepackage[margin = 1.0in]{geometry}
\usepackage{bbold, soul}
\usepackage{tgpagella}
\usepackage{braket, setspace, parskip, enumitem, titling}
\usepackage[colorlinks, citecolor = blue]{hyperref}
\usepackage[nameinlink, noabbrev]{cleveref}	% Use \Cref always 
\usepackage[justification=centering]{caption}
\usepackage{threeparttable, makecell, cellspace, multirow, booktabs, diagbox, tabularx, longtable, threeparttablex, siunitx, rotating}
%\usepackage{verbatim}
\usepackage{natbib}
\usepackage[bottom]{footmisc}	% keep footnotes at end of the page
\usepackage{xcolor, easyReview, pifont}

\newcolumntype{Y}{>{\centering\arraybackslash}X}

\renewcommand\theadfont{\bfseries}
\renewcommand\theadgape{\Gape[4pt]}
\renewcommand\cellgape{\Gape[4pt]}


%\setlist[enumerate]{label=(\roman*)}

%\setlength{\parindent}{0pt}

%\captionsetup[figure]{labelfont=bf}
\captionsetup[table]{labelfont=bf}

%\interfootnotelinepenalty=10000


%opening
\pretitle{\vspace{2cm}\begin{center}\huge}
\title{Maternal risk preference \\and investment in children}
\posttitle{\par\end{center}\vspace{2cm}}
\preauthor{\vspace{1.5cm}\begin{center}\LARGE}
\author{Chau Pham}
\postauthor{\par\end{center}\vspace{2cm}}
\date{}


\newcommand\itembox{\item[$\square$]}
\newcommand{\cmark}{\ding{51}}
\newcommand{\done}{\rlap{$\square$}{\raisebox{2pt}{\large\hspace{1pt}\cmark}}%
	\hspace{-2.5pt}}

\makeatletter
\renewenvironment{abstract}{%
	\if@twocolumn
	\section*{\abstractname}%
	\else %% <- here I've removed \small
	\begin{center}%
		{\bfseries \large\abstractname\vspace{\z@}}%  %% <- here I've added \Large
	\end{center}%
	\quotation
	\fi}
{\if@twocolumn\else\endquotation\fi}
\makeatother



\begin{document}

\onehalfspacing
\begin{titlingpage}
\maketitle
\begin{center}
	\Large Dissertation Supervisor: Professor Pedro Carneiro
\end{center}
\vspace{4cm}
\begin{center}
	\large
	Dissertation submitted in part-fulfilment of the MRes in Economics,\\ \vspace{0.5cm}
	UCL\\ \vspace{0.5cm}
	September 2023
	
\end{center}
\end{titlingpage}


\begin{titlingpage}
	\begin{center}
		{\Large Declaration}
		\vskip 1cm
		\begin{minipage}[!h]{.89\linewidth}\large 
			I, Pham Phu Chau, hereby declare that the work presented in this dissertation is my own original work. Where information has been derived from other sources, I confirm that this has been clearly and fully identified and acknowledged. No part of this dissertation contains material previously submitted to the examiners of this or any other University, or any material previously submitted for any other assessment.
			\vskip 2cm
			Signature
			\vskip 2cm
			September 4, 2023
		\end{minipage}
		\vskip 4cm
		\begin{minipage}[!h]{.89\linewidth}
			\begin{center}
				\Large Classification \vskip 0.5cm
			\end{center}
			\large
			This piece of research is primarily (please tick as appropriate):
			\begin{itemize}
				\item[\done] an empirical/econometric study
				\itembox the examination of a theoretical problem
				\itembox a critical analysis of a policy issue
				\itembox an analytical survey of empirical and/or theoretical literature
			\end{itemize}
		\end{minipage}
	\end{center}
\end{titlingpage}




\begin{titlingpage}
\paragraph*{\vspace{3cm}}
\begin{abstract}\large
	\noindent 
	This article studies the effect of maternal risk preference on investment in children. Using NLSY79, we link risk preference of mothers, measured by the sequence of responses to the lifetime income gamble, to their children and the investment made, measured by various parental inputs and the home environment. Our results indicate a large and negative effect of maternal risk aversion on goods investment for children between 3-5 and 10-14 whose mothers identify as entrepreneurs. We conjecture that these mothers perceive said periods of development of children more ``risky" while treating investment in goods the same as investment in business. The more risk averse mothers, thus, reduce goods input. The effects are much less pronounced and most are found to be statistically insignificant in other dimensions of parental inputs.
\end{abstract}

\vspace{5cm}
\begin{center}
	\begin{minipage}[!h]{.89\linewidth}\small
			\begin{center}
				\large	\textbf{Acknowledgement}
				\end{center} \large
			\noindent I would like to offer my utmost gratitude to my supervisor, Professor Pedro Carneiro, for his valuable guidance and suggestions on the direction and minute details of the dissertation.
		\end{minipage}
\end{center}

\end{titlingpage}

%===================================================================================================================================================================
%===================================================================================================================================================================

\section{Introduction}
Investment in human capital has been considered risky since the works of \citet{becker1962investment, levhari1974effect} and \citet{schultz1971investment}. Returns to investment in human capital, ability, probability of getting a good job are few of the various accounts of risk. The traditional approach on the topic examines how an individual's risk preference affects her investment decision. Earlier theoretical works consider the investment choice in various contexts including migration \citep{david1974fortune}, job mobility \citep{johnson1978theory}, hazardous job \citep{thaler1976value}, and between physical and human capital \citep{levhari1974effect}. 

This line of inquiry is, however, valid insofar as the decision maker is the individual under study. Occupational choice, tertiary education, specific skills training, to name a few, fall under the domain of individual choice wherein an individual is the decision maker and she exercises scrutiny before making her choice. For children and teenagers, the choice are, more often than not, not theirs to make as they lack the appropriate capacity to make an informed choice while various factors such as pecuniary issues are beyond their control. At the same  time, a vast literature on early childhood intervention has deemed these early investments substantially beneficial for future outcomes of children.\footnote{See \citet{Currie2001} and \citet{NORES2010} for a comprehensive review on a large number of childhood intervention programs. Prenatal and postnatal investments have also been found to affect later life outcomes \citep{currie2011human, almond2011killing, currie2013early}.} Furthermore, as highlighted by \citet{jacob2008improving} and \citet{duflo2011poor}, early childhood interventions are most effective in levelling the playground for disadvantaged children. Thus, parents' investment during early childhood plays a pivotal role in shaping the children's future paths. If investment in human capital is risky, the importance of understanding the effect of parental risk preference on children investment naturally follows. 
    

Using National Longitudinal Study of Youth 1979 (NLSY79) and the accompanying dataset on children to mothers in NLSY79, we study the effects of maternal risk aversion on 5 measure parental inputs captured by various aspects of home environment including the general HOME (Home Observation Measurement of the Environment) index, Goods index, Time index, Cognitive Stimulation percentile score and Emotional Support percentile score for a sample of 5167 children. We measure the first three dimensions  Maternal risk aversion is measured by a series of questions on lifetime income gamble administered to participants in 1993, 2002, 2004, 2006, 2010, 2012 and 2014. In each wave, the sequence of responses categorises respondents into four groups, ranked from the least risk averse to the most risk averse. We then follow \citet{kimball2008imputing, kimball2009risk} to deal with response errors and potential confounders through the imputation of coefficient of relative risk aversion (RRA). The explicit correction for confounders from survey responses is necessary as mother's risk preference is associated a number of factors that have direct linkage to investment in children or even life outcomes. \citet{schmidt2008risk} found that high levels of risk tolerance were associated with earlier births at young ages, delayed marriage, and delayed fertility at older ages among highly-educated women in the Panel Study of Income Dynamics (PSID). \citet{spivey2010desperation} also confirmed the longer time-to-marriage for risk-loving respondents compared to their risk-averse counterparts using NLSY79 data. Then risk preference of mothers have potential effects on not only direct investment in children but also indirect channels including mother's education, mother's financial stability, and preparation to motherhood among others. Failure to account for these factors predetermined to the investment decision likely generates a downward bias. In addition, the imputation allows the investigation of the effects of maternal risk aversion on investment in children under two assumptions, namely time-invariant risk preference or constant relative risk aversion (CRRA) and time-varying risk preference. To permit a more in-depth discussion, we also look at how the effects of risk preference vary by age-group of children and by measures of entrepreneurship of mothers.
 
Since past investments likely correlate with future investment, especially in the goods dimension, we adopt a 2-step GMM \citep{arellano1991some, blundell1998initial} with orthogonal deviations \citep{arellano1995another} to deal with autocorrelation in dynamic panel. Our results indicate a negative effect of maternal risk aversion on goods investment. The effect is large, standing at -0.46 point (about 0.5 SD) lower for children of mothers who are 1.15 SD more risk averse than the average mother. This result is robust to various specifications and tests across different subsamples. Further investigation reveal that the negative results are largely driven by mothers who identify as entrepreneurs but not necessarily own any business. Investment in goods made by these mothers for children between 3-5 and 10-14 is especially prone to the mothers' degree of risk aversion. We surmise that these mothers perceive substantial ``risks" when their children enter the ``sensitive" periods, defined as periods when formulation of skills are more productive \citep{knudsen2006economic, cunha2007technology}, and accordingly reduce investment the more risk averse they are. This hypothesis hinges on the assumption that these entrepreneur mothers treat investment in goods for children and investment in firms, such as in physical capital, in a synonymous manner, perhaps, due to their pecuniary nature. In an entrepreneurial setting, more risk averse business owners are found to have smaller firms with lower leverage \citep{herranz2015entrepreneurs}. To the extent that entrepreneur mothers' perception of goods investment is similar to that of business investment, they would reduce investment in goods in periods when ``risks" are high, i.e., when children enter pre-school or puberty.  

This article also contributes to the literature on the determinants of investment in children. Naturally, income and timing thereof has received a lot of attention with copious amounts of articles on the topic (See for instance \citet{brooks1997effects, yeung2002money, carneiro2003human,dahl2012impact, carneiro2016partial, attanasio2020human}). Parental education, parents' cognitive ability and household structure have also been shown to affect children outcomes \citep{chevalier2004parental, brown2004family, guryan2008parental, schady2011parents}. More relevant to our study, parental risk aversion has also been linked to various educational outcomes of children. The majority reported negative effects of risk aversion on educational outcomes including test scores in the US \citep{brown2012parental}, college enrollment in Italy \citep{checchi2014parents}, cognitive outcomes in Indonesia \citep{hartarto2023parental}. Furthermore, \citet{frempong2021risk} found a positive association between risk aversion and child labour. This, coupled with evidence found by \citet{HEADY2003385} and \citet{bezerra2009impact} suggesting child labour impedes learning, also points to a negative relationship between parental risk aversion and educational outcomes. With regards to various school tracks which offer different curriculums, \citet{wolfel2012parental} confirmed the negative effects of maternal risk aversion on daughters for a sample of German students such that daughters of more risk-averse mothers are more likely to pursue lower secondary track. On the other hand, \citet{leonardi2007parents} found no effect of risk aversion on secondary school track choice for a sample of Italian students.\footnote{German and Italy secondary schools mainly separate curriculums based on later-life goals which can be broadly categorized into vocational-oriented education or traditional pathway to university. Schools in Germany also adopt a recommendation system based on the child's ability and which greatly affects track choice.}   

In general, the literature on the impact of parental risk aversion on direct investment in children is scarce whilst results are inconclusive. In direct relation to our study, \citet{sovero2018risk} uses height-for-age, and BMI-for-age along with other several spending categories to measure investment of Mexican parents. Risk attitudes of parents are elicited using a series of questions with multiple price list structure. Her results indicate a positive linkage between risk averse mothers and their investment on sons, i.e., risk averse mothers invest more on sons than daughters. Using data on Uganda households, \citet{tabetando2019parental} confirms the positive relation between risk aversion and investment, measured by level of educational expenditure and its share of household budget. Although wealthier households tend to be more risk averse, the relationship is reversed for poor households. That is, risk aversion depresses investment in credit constrained households. To the best of our knowledge, our study is the first that looks at how maternal risk preference influences investment at various stages of children's development. While we cannot provide full exposition to whether the effects of maternal risk preference vary at different stages as this calls into question a much debated issue in the risk preference literature which pertains to its stability, our results suggest that if mothers react differently at different periods of investment a negative relationship is found for children in sensitive periods whose mothers identify as entrepreneurs. 

Unfortunately, we are unable to establish causality between maternal risk preference and investment in children due to two threats to our identification. The first pertains to the stability across domains of risk preference and over time. We use the sequence of questions on lifetime income gamble to infer the risk preference of mothers but the issue complicates when it comes to the cross-comparison among lifetime income, business investment, and goods investment for children.  \citet{barseghyan2011risk} rejects the null of stable risk preference for insurance on automobiles and homes. Similarly, \citet{einav2012general} found that only 30\% of their sample make consistent choices when presented with a diverse set of options for insurance coverage and 401(k) portfolios. They further note that preference are less consistent for domains that are too disparate whereas choices in closer domains do exhibit some consistency. The issue of temporal stability further compounds the intricacy involving risk preference. The second threat lies in our empirical strategy, more specifically the assumption of the 2-step GMM method which assumes away the existence of valid instruments outside the covariates. This assumption, if violated, generates bias similar to omitted variable bias in the OLS setting. Although we go to great lengths to control for potential confounders, we cannot be sure that no other sources exist.

The rest of the paper is organised as follows. The next section discusses data sources on risk preference and investment. The third section outlines identification and empirical strategy. Results are presented in the fourth section with further robustness checks in the fifth section. The last section concludes our paper.


%===================================================================================================================================================================
%===================================================================================================================================================================

\section{Data}
I use data of National Longitudinal Survey of Youth 1979 (NLSY79) and NLSY79 Child and Young Adults (CYA) to study the effects of parental risk aversion on children investment. The NLSY79 is a longitudinal project that follows the lives of a sample of American youth born between 1957 and 1964 with the first round of data collection in 1979. The original cohort included 12,686 respondents ages 14-22 when first interviewed but later dropped to 9,964 respondents after the scaling down of the military subsample and the exclusion of a portion of the economically disadvantaged, non-Black and non-Hispanic sample. Data are collected annually from 1979 to 1994 and biannually thereafter. Beginning in 1986, additional information was collected biannually about children born to female NLSY79 respondents, constituting the NLSY79 CYA dataset. The Child and Young Adult contains information on a rich set of assessments on cognitive ability, temperament, motor and social development, behaviour problems, self-competence of the children as well as the quality of their home environment. By linking investment, as measured by The HOME (Home Observation Measurement of the Environment) indices along with its subsection indices in the NLSY79 CYA, and an indicator of risk attitudes generated from a series of questions in the NLSY79, it is possible to study the effect of maternal risk aversion on investment in children.

\subsection{Mothers' risk preference}
To measure attitudes towards risk of mothers,\footnote{Since NLSY79 and NLSY79 CYA are linked by identifying unique pairs of mother-child.} Participants of NLSY79 were asked a series of hypothetical questions on occupational choice. The first question of the series reads:

\begin{center}
	\begin{minipage}[!h]{.9\linewidth}\small
		``...Suppose that you are the only income earner in the family, and you have a good job guaranteed to give you your current (family) income every year for life. You are given the opportunity to take a new and equally good job, with a 50-50 chance that it will double your (family) income and a 50-50 chance that it will cut your (family) income by a third. Would you take the new job?"
	\end{minipage}
\end{center}
\begin{flushright}
	\textit{Questionnaire Public Report, National Longitudinal Survey of Youth 1979 (NLSY79), 1993.}
\end{flushright}
Depending on respondents' answers to this question, survey participants were asked a similar question holding the first choice constant while the income cut in the second choice is modified. If participants choose ``Yes" in the first question, the income cut in the second question is lowered to 20\%. If ``No" is chosen in the first question, the income cut is increased to 1/2 in the second question. The first series of questions on risk attitudes were asked in four rounds in 1993, 2002, 2004 and 2006. The new series of questions on lifetime income gamble were asked in 2010. Those who did not participate in 2010 were followed in either 2012 or 2014.\footnote{Unlike in the previous series where multiple answers across time are available for most individuals, responses to the new series are only recorded once per individuals.} The new series begins with the following question:
\begin{center}
	\begin{minipage}[!h]{.9\linewidth}\small
		``Suppose that you are the only income earner in the family, and that you have to choose between two new jobs. The first job would guarantee your current total family income for life. The second job is possibly better paying, but the income is also less certain. There is a 50-50 chance the second job would increase your total lifetime income by 20 percent and a 50-50 change that is would cut it by 10 percent. Which job would you take: the first job or the second job?"
	\end{minipage}
\end{center}
\begin{flushright}
	\textit{Questionnaire Public Report, National Longitudinal Survey of Youth 1979 (NLSY79), 2010.}
\end{flushright} 
Choosing ``First job" would lead respondents to a second question with income cut increased to 15\% as opposed to 5\% for ``Second job". Based on answers to the either series of questions, respondents can be categorized into four groups numbered 1-4 where individuals in group 1 choose ``Yes" in both questions of the old series or "The first job" in both questions of the new series; while individuals in group 4 opt for ``No" in both questions of the old series or "The second job" in both questions of the new series. This means that it is possible for an individual to be placed into different groups in different years, e.g., group 1 in 1993, group 2 in 2002, and group 1 in 2004. By contrast, it is also possible for any individual to stay in one category throughout all the years in which the responses are collected, for instance, she stays in group 1 in all the years she participated in the survey, including 1993, 2006, and 2010. Specifics of groups categorization are presented in \Cref{table:risk-category}. 

\begin{table}[!b]
	\centering
	\setlength{\extrarowheight}{0.2em}
	\caption{Category of risk attitudes}	
	{
\begin{threeparttable}
	\begin{tabular}{|c |l| *{3}{c|}}
		\hline
		\thead{Response\\ category} &\diagbox{\thead{Risk aversion}}{\thead{Question}} & \thead{Original} & \thead{Increased\\ income cut} & \thead{Reduced\\ income cut} \\ \hline
		1 & High risk aversion & First job & - & First job\\
		2 & Moderate risk aversion & First job & - & Second job\\
		3 & Low risk aversion & Second job & First job & -\\
		4 & Lowest risk aversion & Second job & Second job & - \\
		\hline
	\end{tabular}
	\begin{tablenotes}[flushleft]\footnotesize
		\item The table describes response categorisation based on answers provided by participants. This applies to both versions of the questions and year-on-year categorisations are independent of one another.
	\end{tablenotes}
\end{threeparttable}

}
	\label{table:risk-category}
\end{table}

It is noteworthy to point out the difference both in the stakes individuals face in two versions of the questions as well as the way in which the questions are phrased. In the first series, individuals face much higher stakes with the lowest being 20\% of income loss while the highest percentage of income loss in the second series is 15\%. Further, as noted by \citet{kimball2008imputing,kimball2009risk}, the original series presuppose that individuals already have the stable job, i.e., a job without any change to income. This supposition is absent in the newer version wherein individuals are present with two jobs, one stable and one risky. The difference in situations faced by individuals, albeit subtle, may induce the status quo bias. That is, individuals may be more inclined to choose the stable job in the first series not because they are more risk averse but due to their reluctance to change \citep{samuelson1988status,tversky1991loss}. Consequently, having responses from individuals who answered both type of questions aids in the separation of the status quo bias from the true risk preference.













\subsection{Investment data}
Traditionally, investment is often associated with monetary expenditures or goods investment \citep{heckman2014economics}. Only in recent studies such as \citet{carneiro2016partial}, \citet{bono2016early} and \citet{del2016transfers} has time input received more attention as a component of parental input. Since more educated parents spend more quality time with their children despite the higher opportunity cost, time inputs plays a pivotal role in shaping children's future \citep{guryan2008parental}. Aside time input, a nurturing home environment has also been proven conducive to early cognitive development \citep{altman2013home, iverson1982home}. 

\begin{table}[!t]
	\centering
	\setlength{\extrarowheight}{0.2em}
	\caption{Components of HOME index}	
	{
\begin{threeparttable}
	\begin{tabularx}{.97\linewidth}{|l| *{4}{Y|}}
		\hline 
		\diagbox[width = .683\linewidth]{Items}{Age group} & 0-2 & 3-5 & 6-9 & 10-14 \\
		\hline
		\textit{Goods} & & & & \\
		\hspace{3mm} Child has 10 or more soft toys at home & x & & & \\
		\hspace{3mm} Child has 10 or more push/pull toys at home & x & & & \\
		\hspace{3mm} Child has more than 10 books at home & x & x & x & x \\
		\hspace{3mm} Family gets at least three magazines regularly & & x & & \\
		\hspace{3mm} Child has a  CD player & & x & & \\
		\hspace{3mm} Family subscribes to daily newspaper & & & x & x \\
		\hspace{3mm} Child has a musical instrument & & & x & x \\
		& & & & \\
		\textit{Time} & & & & \\
		\hspace{3mm} Child taken to grocery at least once a week & x & & & \\
		\hspace{3mm} Child goes on outings more than three times per month & x & x & & \\
		\hspace{3mm} Child east at least one meal per day with both parents & x & x & x & x \\
		\hspace{3mm} Child sees father(-figure) daily & x & x & x & x \\
		\hspace{3mm} Mother reads to child at least once a week & x & x & x & \\
		\hspace{3mm} Child goes to museum more than twice in past year & & x & x & x \\
		\hspace{3mm} Child spends time with father(-figure) at least four times a week &  &  & x & x \\
		\hspace{3mm} Family gets together with friends/relatives at least twice a month & & & x & x \\
		\hspace{3mm} Child spends time with father(-figure) outdoor once a week & & & x & x \\
		\hspace{3mm} Mother discusses TV programmes with child & & & x & x \\
		\hspace{3mm} Child goes to theatre/performance more than twice in past year & & & x & x \\
		& & & & \\
		\textit{Activities last month\tnote{a}:} & & & & \\
		\hspace{3mm} (NH) Child went shopping with parents & & & & x \\
		\hspace{3mm} (NH) Child went on an outing with parents & & & & x \\
		\hspace{3mm} (NH) Child went with parents to movies & & & & x \\
		\hspace{3mm} (NH) Child went with parents to dinner& & & & x \\
		& & & & \\
		\textit{Activities last week\tnote{a}:} & & & & \\
		\hspace{3mm} (NH) Child worked with parents on schoolwork & & & & x  \\
		\hspace{3mm} (NH) Child did things together with parents & & & & x  \\	
		\hspace{3mm} (NH) Child play game or sports together with parents & & & & x  \\
		\hline			
	\end{tabularx}
	\begin{tablenotes}[flushleft]\footnotesize
		\item \textit{Note: }The table reports items that are used to construct indices of investment for four different age groups: 0-2, 3-5, 6-9, and 10-14. Items under panel Goods are used to construct Goods index for all present children while items under Time are used to construct Time index only for children under 10. Extra items gathered under the last two panels are also used to construct the generic HOME index and Time index for children between 10-14.
		\item[a] Items prefixed by (NH) are not from HOME but from self-administered survey of children.
	\end{tablenotes}
\end{threeparttable}
}
	\label{table:HOME}
\end{table} 

Following \citet{carneiro2016partial}, parental inputs are captured by various elements of HOME (Home Observation Measurement of the Environment). HOME (Short Form) comprises primary measures of the quality of a child's home environment included in the NLSY79 Child survey. Various components of the HOME evaluate the cognitive stimulation and emotional support children below 15 years old receive from the home environment. Items that made up the HOME score are grouped into goods input and time input. These are reported in \Cref{table:HOME}. Apart from measures of HOME section, several items included in the self-administered questions indicating parent-child interaction for children between 10-14 are also present and have the (NH) prefix. These extra items are also used when constructing HOME index and Time index for children within said age-group.  
 

For each age group, to generate the generic investment index (HOME index), indicators of all items in \Cref{table:HOME} for the corresponding age are totalled then normalised to have a mean zero and unit standard deviation.\footnote{The recoding of items follows the procedure used by Bureau of Labor Statistics (BLS) when constructing HOME score (See \citet{center2004nlsy79})} A similar procedure is also applied to items denoting goods related investment and time related investment to generate indices representing the respective type of parental inputs. Other assessments of cognitive stimulation and emotional support are aggregated into percentile score for the respective quality of home. Unlike previous indices, the percentile scores for cognitive  stimulation and emotional support are taken from the NLSY79 CYA dataset.

Now we discuss the timing of investment. Unlike the questions on lifetime gamble which, to some extent, captures an individual's risk preference at the time of survey, the HOME section of the Child \& Young Adult survey contains a number of items that are measured at different time-interval as can be seen from the Time panel in \Cref{table:HOME}. The time-interval can range from daily basis, to weekly or monthly bases. Two items even measure the frequency of occurrence in the previous year. The incongruence of time diary makes comparison and aggregation of items difficult. To allow a more coherent interpretation of index collecting time-related items, we will assume that there are no stark changes in items that are recorded on bases spanning shorter than a year within the year when data is collected.


\subsection{Summary statistics}

\begin{table}[!b]\centering \caption{Summary statistics \label{table:summary}}
	\setlength{\extrarowheight}{0.3em}
	\begin{threeparttable}
		\begin{table}[htbp]\centering \caption{Summary statistics \label{table:5-summary}} \begin{tabular}{l*{1}{cc}} \toprule
                    &\multicolumn{2}{c}{}     \\
\midrule
IMPUTED\_CRRA\_1      &        1.42&      (0.91)\\
IMPUTED\_CRRA\_4      &        1.15&      (0.77)\\
HOME index          &        0.11&      (0.96)\\
Goods index         &        0.08&      (0.98)\\
Time index          &        0.08&      (0.97)\\
Cognitive Stimulation score (\%)&       48.88&     (29.11)\\
Emotional Support score (\%)&       49.95&     (29.23)\\
Number of siblings  &        1.61&      (1.23)\\
Family size         &        4.30&      (1.39)\\
$\#$ weeks worked last year&       32.17&     (22.59)\\
$\#$ weeks spouse worked last year&       48.55&      (9.75)\\
Net household income (\$)&    50883.68&  (78376.30)\\
\bottomrule
\end{tabular}
\end{table}
\begin{tabular}{l*{1}{cc}} \hline\hline
                    &\multicolumn{2}{c}{}     \\
\hline
Female              &        0.49&      (0.50)\\
Black               &        0.25&      (0.43)\\
Non-Black, non-Hispanic&        0.54&      (0.50)\\
Age of mother at child's birth&       27.61&      (5.60)\\
Age of mother at first birth&       23.67&      (5.39)\\
Mother's AFQT score (2006)&       42.95&     (28.52)\\
Mother lives with both biological parents at age 14&        0.74&      (0.44)\\
Mother's mother's years of schooling&       10.73&      (3.28)\\
Mother's father's years of schooling&       10.63&      (4.10)\\
\hline\hline
\end{tabular}

	\begin{tablenotes}[flushleft]\footnotesize
		\item Mean and standard deviations are reported for all numerical variables. Only frequency is reported for indicator variables. The sample is restricted to observations with a full set of controls of child's characteristics and mother's characteristics.
	\end{tablenotes}
	\end{threeparttable}
\end{table}

Descriptive statistics of variables used in analysis are shown in \Cref{table:summary}. The sample is restricted to observations in the year 1994, 2002, 2004, 2006, 2010, 2012 and 2014 to include a measure of risk aversion and which have valid responses (non-missing and non-negative) to child's characteristics, mother's characteristics, and parental inputs.\footnote{Risk preference in year 1993 will be recoded as 1994 to match the closest available investment data. 1994 is chosen instead of 1992 since various items in indices of parental inputs indicate frequency of occurrences in the previous year. Additionally, \citet{hartog2002linking} remarked that risk aversion should be measured before individuals make actual decisions.} This leaves us with 6,905 observations for 5,167 children. The table reports the number of observations, mean, and standard deviations of numerical variables and only frequency for categorical variables. The table is organised into four panels: (1) Imputed values of Relative Risk Aversion (RRA) coefficient $\gamma$ which we will discuss in detail momentarily; (2) parental inputs which include various indices described above (HOME, Goods, Time, Cognitive Stimulation, and Emotional Support); (3) Child's characteristics which include sex, race, age of mother at birth of child, number of siblings and family size; and (3) Mother's characteristics which include an indicator variable showing whether mother lived with both biological parents at age 14, age of mother at first birth, mother's AFQT score (2006 revised edition), years of schooling of mother's parents, indicator variables indicating mother's educational level,\footnote{Mother's education is broadly categorised into three groups: high school dropout, high school and college. Mothers who are in the college group received at least some college education but not necessarily completed college.} number of weeks mother and her partner worked in the previous calendar year, and net household income in the previous calendar year. 

Here we let number of siblings and family size vary with time as these crucially affect the amount of parent-child interaction as well as goods investment. For instance, a child might have received more inputs in the period prior to her sibling(s) being born. It is also possible that parents anticipate the number of children they will have and thus know how their allocation of inputs will shift when all children have been born. In such a narrative, they can compensate the child or children born earlier in anticipation of diverted attention and resources to other later-born sibling(s). Mothers' education is also allowed to vary over time to reflect child's birth in the middle of their studies. 

The average HOME score in our sample is 0.39 which is higher than the average HOME score for the nationally representative sample. A potential cause for this is the restriction on the years in which data chosen which are mostly after 2000 except for the first wave. By the mentioned years, mothers are typically older than the average age of mothers in NLSY79 CYA. This is apparent in the high mean age of mother at child's birth, which is close to 28 in our sample compared to around 25 in the whole sample as reported in \citet{carneiro2016partial}. Higher age also correlates with higher earnings and experience, thus affecting our measures of parental inputs. 

\begin{sidewaystable}[!tbp]
	\def\sym#1{\ifmmode^{#1}\else\(^{#1}\)\fi}
	\setlength{\extrarowheight}{0.3em}
	\begin{threeparttable}
		\caption{Correlation matrix}
		{
\begin{tabular}{l*{7}{c}}
\hline\hline

          &\thead{Time-variant \\$\gamma_v$}         &\thead{Time-invariant \\$\gamma_c$}         &  \textbf{HOME index}         & \textbf{Time index}         &\textbf{Goods index}         &\thead{Emotional Support\\ percentile score (\%)}        &\thead{Cognitive Stimulation\\ percentile score (\%)}        \\
\hline
Time-variant $\gamma_v$&        1         &                  &                  &                  &                  &                  &                  \\
Time-invariant $\gamma_c$&    0.891\sym{***}&        1         &                  &                  &                  &                  &                  \\
HOME index   &  -0.0181         &-0.000995         &        1         &                  &                  &                  &                  \\
Time index  &  0.00439         &  0.00458         &    0.886\sym{***}&        1         &                  &                  &                  \\
Goods index &  -0.0500\sym{***}& -0.00682         &    0.664\sym{***}&    0.259\sym{***}&        1         &                  &                  \\
\makecell[l]{Emotional Support\\ percentile score (\%)} &  -0.0512\sym{***}&  -0.0384\sym{**} &    0.354\sym{***}&    0.358\sym{***}&    0.171\sym{***}&        1         &                  \\
\makecell[l]{Cognitive Stimulation\\ percentile score (\%)}&  0.00227         &   0.0188         &    0.574\sym{***}&    0.413\sym{***}&    0.582\sym{***}&    0.254\sym{***}&        1         \\
\hline\hline
\end{tabular}
}

		\begin{tablenotes}[flushleft]\footnotesize
			\item The table reports pairwise correlation of imputed values of RRA and parental inputs for the restricted sample with a full set of control of mother and child' characteristics.
			\item Significance level following $p$-value calculated from t-test for pairwise correlation: \sym{*} \(p<0.05\), \sym{**} \(p<0.01\), \sym{***} \(p<0.001\).  
		\end{tablenotes}
		\label{table:correlation-matrix}
	\end{threeparttable}
\end{sidewaystable}

The correlation matrix of Imputed values of RRA and Parental inputs are presented in \Cref{table:correlation-matrix}. It also reports statistical significance based on the t-test for pairwise correlation. The correlations between imputed values of RRA and Emotional Support Score are found to be statistically significant under both assumptions whereas Goods index is only robustly correlated with the time-variant version of imputed RRA. At this point, while there is co-movement among these variables that is not due to pure chance, it is difficult to tell whether these correlations imply causations. We expect the effect of risk preference on parental inputs to be confounded by a number of factors including educational level, cognitive ability, age at child's birth among others. The next section discusses how we deal with these potential confounders.   



%===================================================================================================================================
%===================================================================================================================================


\section{Methodology}
\subsection{The choice of proxy for risk}
Several issues arise with the use of the series of questions on lifetime income gamble to capture risk preference of mothers when she make investment on her children, namely relevance, stability and temporal properties. 

First, it is crucial to examine the suitability of this measure in capturing true risk attitudes as highlighted by \citet{yilmazer2015portfolio}. To this end, note that the question wordings and structure of the questions on risk preference in the NLSY79 are the same as those in the Health and Retirement Study (HRS) as well as the Panel Study of Income Dynamics (PSID). While the method of eliciting risk preference using lifetime income is uncommon and most other methods fall into the category of self-evaluated risk willingness/tolerance employing Likert scale, gamble choice \citep{eckel2002sex}, or multiple price list (MPL),\footnote{For a comprehensive review on methods of elicitation of risk preference, see \citet{charness2013experimental}} \citet{barsky1997preference} showed that this measure does predict an array of risky behaviours including drinking, smoking, and holding of risky assets. Furthermore, gamble on hypothetical choice in general has been shown to perform relatively well compared to its counterpart which offer real monetary rewards \citep{binswanger1981attitudes, camerer1999effects, dohmen2005individual}. Given that these series of questions have predictive power over several risky behaviours as well as engagement in activities involving risks such as holding risky assets and participation in the financial market, the measure should be an appropriate proxy for true risk preference. 

The second concern is whether the two versions of the questions on lifetime income gamble provide a consistent proxy for risk preference. More elaborately, it begs the question ``Besides the status quo bias stemming from different phrasing of the questions, do mothers change their risk preference due to different stakes involved?" This problem is usually referred to in the literature as the stability or consistency of risk preference across contexts. At large, this literature explores how revealed preference changes with different elicitation methods, phrasing of questions, the amount of stakes, and the domains in which risks manifest. In our case, respondents are administered two hypothetical gambles on lifetime income with different stakes. As shown by \citet{holt2002risk}, there are shifts in risk preference when participants consider gambles with varying magnitude of stakes when they involve real monetary payoffs. However, no such change in risk preference is found when the stakes are hypothetical. Consequently, responses to hypothetical lifetime income gamble can provide a consistent proxy for risk preference.


The third issue that we will address is the temporal properties of risk preference. We pay close attention to this on the following grounds. On the one hand, if risk preference is static and time-invariant, changes in investment in children across time are independent of risk preference. On the other hand, if risk preference varies over time, to distinguish the impact of risk preference from potential confounders, we need not only account for potential endogeneity and confounding factors, but also capture the dynamic linkage between risk attitudes and investment in children. \citet{sahm2012much} found that systematic differences in risk preference across individuals and over time mostly stem from time-constant characteristics such as gender and ethnicity. To a lesser extent, aging and fluctuations in macroeconomic conditions also contribute to these discrepancies while the effects of the latter is short-lived. Surprisingly, individual circumstances such as job loss, divorce or serious health events do not alter risk preference. \citet{gorlitz2020parenthood} further found that while parenthood considerably affects risk aversion, that is risk aversion increases as early as 2 years prior to becoming a parent, risky labour market behaviour remains unaffected by parenthood. As our measure of risk is closely linked to labour market behaviour, it can capture risk preference without being affected by parenthood. To further separate true risk preference from underlying systemic factors, we will use 2-steps imputation methods by \citet{kimball2008imputing,kimball2009risk} further explained below.   



\subsection{Identification of risk preference}

Using the series of questions on risk attitudes, following \citet{kimball2008imputing,kimball2009risk}, under the assumption that respondents have constant relative risk aversion (CRRA) and risk tolerance are log-normally distributed, we calculated and imputed values of CRRA coefficient for each respondent. The advantages of a cardinal proxy as opposed to other ordinal measures are two-fold. First, it provides a unidimensional, quantitative measure of risk preference that allows meaningful interpersonal comparisons. Second, having a quantitative measure makes the correction of measurement error associated with survey responses explicit and transparent. Assuming that CRRA well approximates an individual's utility over lifetime income,
\[U(W)={W^{1-{1/\theta}}\over 1-{1/\theta}}\]
where $\theta$ is the coefficient of relative risk tolerance and its reciprocal, $\gamma = 1/\theta$, coefficient of relative risk aversion, bounds on $\theta$ are calculated and reported in \Cref{table:risk-bound} for each category. 

\begin{table}[!b]
	\centering
	\setlength{\extrarowheight}{0.3em}
	\caption{Risk tolerance response category}	
	{
\begin{threeparttable}
	\begin{tabular}{*{7}{c}}
		\toprule
		\multirow{2}{*}{\thead{Response\\ category}}& \multicolumn{2}{c}{\thead{Downside risk of\\ risky jobs}} & & \multicolumn{2}{c}{\thead{Bounds on \\risk tolerance}} \\
		\cmidrule{2-3} \cmidrule{5-6}
		& Accepted & Rejected & & Lower & Upper \\
		\hline
		1 &  None & 20\% && 0 & 0.27 \\
		2 &  20\% & 1/3 && 0.27 & 0.5 \\
		3 &  1/3 & 1/2 && 0.5 & 1 \\
		4 &  1/2 & None && 1 & $\infty$ \\
		\bottomrule
	\end{tabular}
	\begin{tablenotes}[flushleft] \footnotesize
		\item The table reports the implied downside risks accepted or rejected based on response category of NLSY79 respondents. Under the assumption of CRRA, the bounds on coefficient of risk tolerance $\theta$ based on the response category are calculated and present on columns 4 and 5.
	\end{tablenotes}
\end{threeparttable}

}
	\label{table:risk-bound}
\end{table}

To derive the main imputations of HRS, \citet{kimball2008imputing} adopts the structural assumption that risk preference is time-invariant. Differences in responses of an individual across waves to the same type of questions are then attributed to response errors which are further decomposed into transitory error and persistent error, both normally distributed with zero mean. However, according to \citet{dohmen2011individual} and \citet{dohmen2017risk}, willingness to take risks varies over the course of life-cycle and trends downward as individuals age. These evidences are consistent with \citet{sahm2012much}. To test for potential dynamic relationship between maternal risk preference and investment in children, we estimate the models with imputations under the assumptions of constant risk preference as well as floating risk preference. The former assumption impose a stringent constraint on risk preference such that risk preference can only depend on survey responses to the lifetime income gamble. The latter assumption is more flexible as it allows for risk preference to, conditioned on survey responses, not only fluctuate with time but also with other observable characteristics of an individuals.  

The procedure entails 2 steps. First, for each wave that collects responses to risk preference questions, the status quo bias, mean and variance of log risk tolerance, and variance of transitory and persistent response error are estimated using maximum likelihood.\footnote{For clarity, only those who responded to both questions in the risk preference series are included. Individuals who only provide answer to the first question are excluded from the sample. To circumvent any biases potentially introduced by sample selection, for each wave, MLE is carried out on the entirety of the NLSY79 dataset. Likelihood functions and results of the first-stage MLE are reported in \alert{Appendix A}.} Having obtained estimates of the mean and log risk tolerance in the first stage, we impute the coefficient of risk aversion, $\gamma$, via conditional expectation that an individual belongs to a certain risk category $j$. Depending on the assumptions imposed, two patterns can be observed. Under the assumption of time-constant risk preference, the imputations are constant for an individual for a risk category across years. \alert{If an individual are grouped into different categories in different years, she can still be assigned different values of imputed $\gamma$ but for years in which she belong to the same category, the assigned RRA values must be the same.} If risk preference is time-varying, an individual can have varying degree of risk aversion across waves even if she belongs to the same category.\footnote{While there are evidences suggesting linear increase in risk aversion over the life-cycle \citep{dohmen2017risk}, individuals do not necessarily have a priori increasing risk aversion over years as there are also supporting evidence on the increase in willingness to take risk when entrepreneurs accumulate more experience  \citep{brachert2017simultaneity,cho2021endogenous}. We allow risk aversion to float freely with other observable factors further described in \alert{Appendix A}.} \Cref{table:CRRA-summary} reports the means and standard deviations of the imputed RRA values under the two assumptions within the NLSY79 dataset and the restricted sample of mothers used for linking with investment in children.

\begin{table}
	\centering
	\setlength{\extrarowheight}{0.3em}
	\begin{threeparttable}
		\caption{Summary statistics of imputed RRA coefficients}	
		\begin{table}[htbp]\centering \caption{Summary statistics of imputed CRRA} \begin{tabular}{l*{8}{c}} \hline\hline
                    &\multicolumn{8}{c}{}                                                                                   \\
                    &        1994&        2002&        2004&        2006&        2010&        2012&        2014&       Total\\
\hline
IMPUTED\_CRRA\_1      &        1.55&        1.53&        1.52&        1.53&        1.53&        1.70&        1.55&        1.53\\
                    &      (1.02)&      (1.05)&      (1.07)&      (1.08)&      (1.06)&      (1.04)&      (0.90)&      (1.05)\\
[1em]
IMPUTED\_CRRA\_4      &        1.68&        1.08&        1.12&        1.09&        1.57&        1.15&        1.03&        1.32\\
                    &      (1.17)&      (0.78)&      (0.80)&      (0.78)&      (1.11)&      (0.69)&      (0.61)&      (0.99)\\
\hline
Observations        &       38999&            &            &            &            &            &            &            \\
\hline\hline
\end{tabular}
\end{table}
\begin{table}[htbp]\centering \caption{Summary statistics of imputed CRRA} \begin{tabular}{l*{8}{c}} \hline\hline
                    &\multicolumn{8}{c}{}                                                                                   \\
                    &        1994&        2002&        2004&        2006&        2010&        2012&        2014&       Total\\
\hline
IMPUTED\_CRRA\_1      &        1.42&        1.43&        1.44&        1.47&        1.51&        1.48&        0.83&        1.44\\
                    &      (0.89)&      (0.95)&      (0.96)&      (0.98)&      (0.99)&      (0.48)&      (0.07)&      (0.93)\\
[1em]
IMPUTED\_CRRA\_4      &        1.38&        0.97&        0.98&        0.97&        1.43&        0.88&        0.44&        1.17\\
                    &      (0.89)&      (0.66)&      (0.65)&      (0.65)&      (0.96)&      (0.25)&      (0.04)&      (0.81)\\
\hline
Observations        &       14244&            &            &            &            &            &            &            \\
\hline\hline
\end{tabular}
\end{table}

		\label{table:CRRA-summary}
	\begin{tablenotes}[flushleft]\footnotesize
		\item The table reports the means, standard deviations and number of observations of the imputed RRA coefficients by year, under assumptions of time-constant or time-varying, and in the full NLSY79 dataset or the restricted sample of mothers with available data on investment in children and a full set of control described in \Cref{table:summary}.
	\end{tablenotes}
	\end{threeparttable}
\end{table}    



It is readily apparent that the imputed values for the whole dataset is higher than the restricted sample. A potential source of this salient difference is the well-established findings in the literature that females are more risk averse than males (See, for instance, \citet{croson2009gender, jianakoplos1998women, powell1997gender}). Under the assumption of floating risk preference, gender difference in risk preference is captured by the observable characteristic sex and after separating this systemic difference, women in our sample are found to be less risk-averse than men.\footnote{Although panel B of the table shows the mean and standard deviation for the restricted sample of mothers with a full set of controls, the statement still hold when we weaken the restriction to include all mothers with available investment data.} This is a debatable area in the literature studying risk preference since risk preference is context-dependent and no definitive results have been found to corroborate or negate the statement. If we assume that women and men are equivalent in terms of risk preference if not for their sex, a potential cause for the downward bias in the imputed values might be the upward bias of the effect of sex on risk preference. That is, differences in sex may have far-reaching impact on other aspects that are also relevant to risk preference and which feed into this upward bias.  On the other hand, under the assumption of time-constant risk preference, a similar mechanism exaggerates the persistent response error of the female sample. That being the case, we do not investigate this issue further since it is not the main focus of our study.  

Another striking point from \Cref{table:CRRA-summary} is the steady reduction in the number of observations across the year in the restricted sample of mothers with available data on investment in children. This is natural if we take into account the age of mothers in our sample in the corresponding years. In 1993, the first wave of question on risk preference, mothers were between the age of 28 and 36. In 2010, they were between 45 and 53. Since our sample only include children below 15, having data on investment in the year 2010 means that mothers were at least 30 at child's birth. This age is remarkably higher than the average age of mother at child's birth as presented in \Cref{table:summary}. Consequently, there are fewer mothers whose children were below 15 in 2010 compared to 1993. 


\subsection{Empirical strategy}
To estimate the effect of risk preference on parental inputs, we follow the 2-step system GMM \citep{arellano1991some,blundell1998initial} with corrected-standard error \citep{windmeijer2005finite} and acknowledge the potential autocorrelation of parental inputs. That is, investment in children in a particular year might correlate with that of previous year(s), especially for investment in goods such as books or toys. Thus, failing to account for autocorrelation in the error term leads to biased estimates, also known as ``dynamic panel bias" \citep{nickell1981biases}. Since our dataset is unbalanced with gaps, taking first difference will exclude a large portion of the sample, i.e., the year 1994, 2010, 2012, and 2014 will be omitted from the estimation due to the absence of responses on risk preference.\footnote{Note that while 2010, 2012 and 2014 are considered continuous in our panel, each participants only responded to the risk preference questions once across all three years. The lack of repeated responses thus make said years unsuitable for estimation in difference.} To circumvent this issue, we adopt the \textit{forward orthogonal deviations} approach introduced by \citet{arellano1995another} which subtracts the average of all future available observations of a variable to deal with endogeneity in short panels and minimise data loss. Formally, we estimate a dynamic model with exogenous variables,
\begin{equation}
	I_{ij,t}=\alpha I_{ij,t-2} + \delta\gamma_{j,t-2}+ \beta_1' Z_{ij,t} + \beta_2' Z_{ij} + \eta_i + \lambda_t + \kappa + v_{ij,t}
	\label{eq:empirical-specification}
\end{equation}
where $I_{ij,t}$ denotes the parental input child $i$ of mother $j$ receives at time $t$. $\gamma_{j,t-2}$ is the imputed RRA coefficient of mother $j$ 2 years prior to the investment time since some investment are recorded with respect to past year and to deal with potential reverse causality, risk preference should be measured before an investment decision is made \citet{hartog2002linking}.\footnote{Since data is recorded biennially, we can only include 2-period lagged explanatory variables instead of 1-period lag variables.} $Z_{ij,t}$ denote child $i$ and mother $j$' time-variant characteristics number of siblings, family size, number of weeks mother and mother's spouse worked in previous year, and household net income in the previous calendar year. $Z_{ij}$ denote child $i$ and mother $j$' time-invariant characteristics including child $i$'s sex, mother's age at child's birth, mother's age at first birth, mother's AFQT score (2006 revised edition), educational level of mother, an indicator variable for mother's residence at age 14, mother's parents education. The specification also contains unobservable child-specific effect $\eta_i$, a year-specific effect common to all children at time $t$, $\lambda_t$, age-group specific effect $\kappa$, and an idiosyncratic shock $v_{ij,t}$ assumed to be purely stochastic.\footnote{We include age-group fixed effect since parental inputs are normalised by age-group, namely 0-2, 3-5, 6-9, and 10-14.} Standard error are clustered at mother's level to allow correlation among siblings. All time-variant variables are assumed to be endogenous while time-invariant exogenous.

Under the assumption of time-constant risk preference, $\gamma_{j,t-2}$ is strictly exogenous, such that $\mathbb E[\gamma_{j,t-2}v_{i,s}]=0$ for all $t,s$. Thus, consistent estimate of $\delta$ in this case can be obtained through the inclusion of $\gamma$ as an instrument for other endogenous variables alongside the exogenous variables. On the other hand, if risk preference is time-variant, $\gamma_{j,t-2}$ is predetermined, in the sense that $\mathbb E[\gamma_{j,t-2}v_{ij,s}]\ne 0$ for $s<t-2$ but zero otherwise. Then $\delta$ can be consistently estimated by also treating $\gamma$ as an instrument for endogenous variables but the set of valid instruments reduce to $\gamma_{j,1},...,\gamma_{j,t-3}$. In practice, however, due to gaps in responses to lifetime income gamble, we treat $\gamma_{j,t-2}$ purely as an endogenous variable.

To have a better sense of why and how risk aversion might affect parental inputs, we also examine how the effects of maternal risk aversion vary by age-group of their children. Furthermore, as it is commonly known that entrepreneurs have higher levels of risk tolerance than their employees counterparts, we also study how the effects of risk preference vary in these two subgroups. To this end, we employ two survey measures and one constructed measure. The first one is an indicator of whether mothers own any type of business.\footnote{This is irrespective of whether the business went into operation before or after the birth of a child.} However, a closer inspection at the data reveals that there are respondents who owned no business while answering "Yes" to the question on business ownership. This motivates us to reclassify respondents by the actual number of business they own, i.e., the indicator takes value 1 if respondents owned at least one business and 0 otherwise. The second survey measure we use is respondents' answers to the question ``Do you consider yourself to be an entrepreneur?" It should be noted that not all those who own business identify as entrepreneurs and not all those who identify as entrepreneurs own at least one business. This measure, thus, captures one's self-identity and possibly her mindset rather than reflect actual business ownership.


%===================================================================================================================================================================
%===================================================================================================================================================================

\section{Results}
We present the estimates of the effect of maternal risk preference on parental input in \Cref{table:main-result}. For comparison, we also include OLS estimates with child-specific effect. \Cref{table:main-result} is organised into 5 panels of each measures of parental inputs, namely HOME index, Goods index, Time index, Cognitive stimulation score (\%) and Emotional support score (\%). For each panel, the first two columns report the OLS estimates with fixed effect of \eqref{eq:empirical-specification}. Since child fixed-effect would collinear with all time-invariant covariates, we instead interact these time-constant characteristics with year and allow their effects to vary across years. The next two columns utilise the two-step system GMM with orthogonal deviation as described in section 3. Both OLS estimates and GMM estimates are reported for two assumptions on the temporal properties of risk preference, i.e., time-constant or time-varying. All models include a full set of controls of child's characteristics (sex, race, age of mother at child's birth, number of siblings, family size) and mother's characteristics (age of mother at first birth, mother's AFQT score, education level of mother, an indicator of whether mother lived with both biological parents at age 14, education levels of mother's parents, number of weeks worked of mother and mother's partner in the previous year, log household income the previous calendar year) as described in \Cref{table:summary}.  


\begin{table}[!t]
	\centering
	\begin{threeparttable}
		\def\sym#1{\ifmmode^{#1}\else\(^{#1}\)\fi}
		\caption{The effect of maternal risk aversion on Parental inputs}	
		\setlength{\extrarowheight}{0.3em}
		{
\begin{tabular}{l*{5}{c}}
\toprule
                &\multicolumn{2}{c}{OLS} && \multicolumn{2}{c}{2-step GMM}\\ \cline{2-3} \cline{5-6}
                	& Time-invariant & Time-variant && Time-invariant & Time-variant \\
\midrule
\addlinespace \multicolumn{6}{l}{\textit{Panel A: HOME index}} \\
CRRA Coefficient: $\gamma$&   -0.433         &     -0.108      &       &    0.003         &       -0.165           \\
                &  (1.256)         &      (0.097)           & &  (0.017)         &       (0.135)           \\
\midrule
Observations    &     5032         &     5032   &      &     5032         &     5032         \\

\midrule

 \multicolumn{6}{l}{\textit{Panel B: Goods index}} \\
CRRA Coefficient: $\gamma$&   -0.428         &      -0.040     &       &    0.006         &     -0.461\sym{**}             \\
                &  (1.361)         &       (0.102)          & &  (0.018)         &        (0.183)          \\
\midrule
Observations    &     5017         &     5017    &     &     5017         &     5017         \\

\midrule

 \multicolumn{6}{l}{\textit{Panel C: Time index}} \\
CRRA Coefficient: $\gamma$&   -0.210         &      -0.139       &     &   -0.002         &         -0.086           \\
                &  (1.224)         &       (0.103)          & &  (0.018)         &         (0.145)         \\
\midrule
Observations    &     5017         &     5017         & &    5017         &     5017         \\

\midrule

 \multicolumn{6}{l}{\textit{Panel D: Cognitive Stimulation percentile score (\%)}} \\
CRRA Coefficient: $\gamma$&  -28.426         &     -4.978  &            &    0.294         &      2.354               \\
                & (64.179)         &       (4.061)          & &  (0.563)         &        (6.669)           \\
\addlinespace
\midrule
Observations    &     4264         &     4264         &   &  4264         &     4264         \\

\midrule

 \multicolumn{6}{l}{\textit{Panel E: Emotional Support percentile score (\%)}} \\
CRRA Coefficient: $\gamma$&  -72.030         &     -6.695  &           &   -1.265\sym{*}  &     14.242               \\
                & (59.987)         &       (4.853)          & &  (0.721)         &       (11.597)            \\
\addlinespace
Observations    &     3816         &     3816         & &    3816         &     3816         \\
\bottomrule
\end{tabular}
}

		\label{table:main-result}
		\begin{tablenotes}[flushleft] \footnotesize
			\item \textit{Note:} The table present OLS estimates and 2-step GMM estimates of risk aversion on parental inputs. The set of controls included in the results but omitted here due to space constraints and ease of representation consist of child characteristics (sex, age of mother at child's birth, number of siblings, family size) and maternal characteristics (age of mother at first birth, AFQT score, indicator of whether mother lived with both biological parents at age 14, educational level of mother's parents, number of weeks worked last year and log of household income last year). All models are augmented with time effect, child-specific effect and age-category fixed effect. All time-invariant variables are interacted with time in the OLS specification. In the 2-step GMM setting, $\gamma$ is treated as an instrument under the assumption of time-invariant risk preference and endogenous otherwise. Standard errors are clustered by mother and shown in parentheses. 
			\item Significant levels: \sym{*} \(p<0.10\), \sym{**} \(p<0.05\), \sym{***} \(p<0.01\).
		\end{tablenotes}
	\end{threeparttable}
\end{table}

No significant effects are found in the OLS specification. Risk aversion is found to have a statistically significant at the 5\% significance level and sizeable negative impact on goods investment under the assumption of time-varying risk preference. Ceteris paribus, a unit increase (about 1.15 SD) in the time-variant RRA coefficient depresses investment in goods by 0.461, slightly more than half a standard deviation of goods index. Even though the effect is considerably large, it is worth noting that a unit increase from the mean of time-variant $\gamma$ of 1.09 push an individual from the 60th percentile to the 90-95th percentile in terms of risk aversion. The leap is even more noticeable for the most risk-tolerant individuals, pushing him from the 1st percentile to the 70-75th percentile. We also found a negative impact of risk aversion on Emotional support score which is statistically significant at the 10\% confidence level under the assumption of time-invariant risk preference. A unit increase in CRRA coefficient is associated with a 1.27 percentage point (less than 0.05 SD) decrease in the Emotional Support score. 

To have a better sense of why these results arise, we now examine how the effect of maternal risk aversion varies by age-group of children. \Cref{table:result-age} repeats the GMM estimation of \eqref{eq:empirical-specification} without $\kappa$ for each age-group. For each panel, the first two lines present point estimates and standard error of $\gamma$ under the assumption of time-invariant risk preference and the next two lines time-variant risk preference. A number of striking results arise from this exercise. As regards goods index, statistically and economically significant results are found for children between 3-5 under both assumptions albeit in opposition to one another, and for children between 10-14 under the assumption of time-variant risk preference. For children between 3 and 5 years of age, a unit increase their mothers' CRRA coefficient is associated with a 0.1 point (about 0.1 SD) increase in goods investment under the assumption of constant risk preference as opposed to a 0.6 point (close to 0.7 SD) drop in goods investment under the assumption of varying risk preference. Risk aversion is also shown to affect children between 10-14 under the latter assumption with a drop of 0.449 point (about 0.5 SD) in goods investment in response to one point increase in RRA coefficient under the latter assumption. As for emotional support, statistically significant and negative effect is found only for children between 0-2 such that a unit increase in the time-invariant CRRA coefficient induces a 12.95 percentage point (slightly less than 0.5 SD) drop in emotional support percentile score.   

\begin{table}[!t]
	\centering
	\begin{threeparttable}
		\def\sym#1{\ifmmode^{#1}\else\(^{#1}\)\fi}
		\caption{The effect of maternal risk aversion on Parental inputs by child's age-group}	
		\setlength{\extrarowheight}{0.3em}
		{
\def\sym#1{\ifmmode^{#1}\else\(^{#1}\)\fi}
\begin{tabular}{l*{4}{c}}
\toprule
                &0-2&3-5&6-9&10-14\\
\midrule
\multicolumn{5}{l}{\textit{Panel A: HOME index}} \\
Time-invariant CRRA: $\gamma_c$&   -0.075         &    0.034         &   -0.001         &    0.016         \\
                &  (0.074)         &  (0.029)         &  (0.023)         &  (0.021)         \\

Time-variant CRRA: $\gamma_v$&   -0.648         &   -0.207         &    0.114         &   -0.101         \\
                &  (0.532)         &  (0.183)         &  (0.231)         &  (0.169)         \\
\midrule
Observations    &      191         &      802         &     1619         &     2420         \\



\midrule
\multicolumn{5}{l}{\textit{Panel B: Goods index}} \\
Time-invariant CRRA: $\gamma_c$&   -0.129         &    \textbf{0.101\sym{***}}&    0.018         &    0.004         \\
                &  (0.093)         &  (0.034)         &  (0.032)         &  (0.019)         \\

Time-variant CRRA: $\gamma_v$&    0.152         &   \textbf{-0.602\sym{***}}&   -0.077         &   \textbf{-0.449\sym{**}} \\
                &  (0.786)         &  (0.225)         &  (0.268)         &  (0.214)         \\
\midrule
Observations    &      191         &      801         &     1618         &     2407         \\




\midrule
\multicolumn{5}{l}{\textit{Panel C: Time index}} \\
Time-invariant CRRA: $\gamma_c$&   -0.018         &   -0.000         &   -0.020         &    0.016         \\
                &  (0.076)         &  (0.031)         &  (0.024)         &  (0.024)         \\
Time-variant CRRA: $\gamma_v$&   -0.929         &   -0.328         &    0.019         &    0.016         \\
                &  (0.921)         &  (0.220)         &  (0.181)         &  (0.229)         \\
\midrule
Observations    &      191         &      801         &     1618         &     2407         \\




\midrule
\multicolumn{5}{l}{\textit{Panel D: Cognitive Stimulation score (\%)}} \\
Time-invariant CRRA: $\gamma_c$&   -4.604         &   -0.982         &    1.138         &    0.241         \\
                &  (3.755)         &  (1.390)         &  (0.966)         &  (0.654)         \\
Time-variant CRRA: $\gamma_v$&   18.525         &  -10.585         &   -5.618         &    0.242         \\
                & (30.124)         &  (8.154)         &  (5.701)         &  (7.164)         \\
\midrule
Observations    &      145         &      599         &     1397         &     2123         \\




\midrule
\multicolumn{5}{l}{\textit{Panel E: Emotional Support score (\%)}} \\
Time-invariant CRRA: $\gamma_c$&  \textbf{-12.954\sym{***}}&   -0.446         &   -0.029         &   -0.949         \\
                &  (3.399)         &  (1.635)         &  (1.076)         &  (1.109)         \\

Time-variant CRRA: $\gamma_v$&   -7.266         &    7.097         &  -12.541         &    3.802         \\
                & (21.421)         &  (9.287)         &  (9.497)         & (11.120)         \\
\midrule
Observations    &      111         &      550         &     1315         &     1840         \\
\bottomrule
\end{tabular}
}

		\label{table:result-age}
		\begin{tablenotes}[flushleft] \footnotesize
			\item \textit{Note:} The table present 2-step system GMM estimates of risk aversion on parental inputs by age-group of children. All specifications include a full set of controls with only child-specific effect and time-specific effect. For each panel, the first two lines present point estimates and standard error of $\gamma$ under the assumption of time-invariant risk preference and the next two lines time-variant risk preference. Standard errors are clustered by mother and shown in parentheses. 
			\item Significant levels: \sym{*} \(p<0.10\), \sym{**} \(p<0.05\), \sym{***} \(p<0.01\).
		\end{tablenotes}
	\end{threeparttable}
\end{table}


\begin{table}[!htbp]
	\centering
	\begin{threeparttable}
		\def\sym#1{\ifmmode^{#1}\else\(^{#1}\)\fi}
		\caption{GMM estimates of risk aversion on Parental inputs by entrepreneurship}	
		\setlength{\extrarowheight}{0.3em}
		{
\def\sym#1{\ifmmode^{#1}\else\(^{#1}\)\fi}
\begin{tabular}{l*{8}{c}}
\toprule
                &\multicolumn{1}{c}{}&\multicolumn{1}{c}{model\_11\_b}&\multicolumn{1}{c}{model\_11\_c}&\multicolumn{1}{c}{model\_11\_d}&\multicolumn{1}{c}{model\_11\_e}&\multicolumn{1}{c}{model\_11\_f}&\multicolumn{1}{c}{model\_11\_g}&\multicolumn{1}{c}{model\_11\_h}\\
\midrule
L.IMPUTED\_CRRA\_1&     0.02         &                  &     0.00         &                  &     0.01         &                  &     0.02         &                  \\
                &   (0.02)         &                  &   (0.03)         &                  &   (0.02)         &                  &   (0.03)         &                  \\
\addlinespace
L.IMPUTED\_CRRA\_4&                  &    -0.36         &                  &     0.13         &                  &    -0.16         &                  &    -0.22         \\
                &                  &   (0.22)         &                  &   (0.21)         &                  &   (0.19)         &                  &   (0.18)         \\
\midrule
Observations    &     4134         &     4134         &      719         &      719         &     4150         &     4150         &      677         &      677         \\
\bottomrule
\multicolumn{9}{l}{\footnotesize Standard errors in parentheses}\\
\multicolumn{9}{l}{\footnotesize \sym{*} \(p<0.10\), \sym{**} \(p<0.05\), \sym{***} \(p<0.01\)}\\
\end{tabular}
}
{
\def\sym#1{\ifmmode^{#1}\else\(^{#1}\)\fi}
\begin{tabular}{l*{8}{c}}
\toprule
                &\multicolumn{1}{c}{}&\multicolumn{1}{c}{model\_12\_b}&\multicolumn{1}{c}{model\_12\_c}&\multicolumn{1}{c}{model\_12\_d}&\multicolumn{1}{c}{model\_12\_e}&\multicolumn{1}{c}{model\_12\_f}&\multicolumn{1}{c}{model\_12\_g}&\multicolumn{1}{c}{model\_12\_h}\\
\midrule
L.IMPUTED\_CRRA\_1&     0.03         &                  &    -0.01         &                  &     0.01         &                  &     0.02         &                  \\
                &   (0.02)         &                  &   (0.03)         &                  &   (0.02)         &                  &   (0.03)         &                  \\
\addlinespace
L.IMPUTED\_CRRA\_4&                  &    -0.30         &                  &     0.11         &                  &    -0.19         &                  &    -0.48\sym{**} \\
                &                  &   (0.23)         &                  &   (0.17)         &                  &   (0.16)         &                  &   (0.23)         \\
\midrule
Observations    &     4120         &     4120         &      718         &      718         &     4137         &     4137         &      675         &      675         \\
\bottomrule
\multicolumn{9}{l}{\footnotesize Standard errors in parentheses}\\
\multicolumn{9}{l}{\footnotesize \sym{*} \(p<0.10\), \sym{**} \(p<0.05\), \sym{***} \(p<0.01\)}\\
\end{tabular}
}
{
\def\sym#1{\ifmmode^{#1}\else\(^{#1}\)\fi}
\begin{tabular}{l*{8}{c}}
\toprule
                &\multicolumn{1}{c}{}&\multicolumn{1}{c}{model\_13\_b}&\multicolumn{1}{c}{model\_13\_c}&\multicolumn{1}{c}{model\_13\_d}&\multicolumn{1}{c}{model\_13\_e}&\multicolumn{1}{c}{model\_13\_f}&\multicolumn{1}{c}{model\_13\_g}&\multicolumn{1}{c}{model\_13\_h}\\
\midrule
L.IMPUTED\_CRRA\_1&     0.01         &                  &     0.01         &                  &    -0.00         &                  &     0.03         &                  \\
                &   (0.02)         &                  &   (0.03)         &                  &   (0.02)         &                  &   (0.03)         &                  \\
\addlinespace
L.IMPUTED\_CRRA\_4&                  &    -0.41         &                  &     0.14         &                  &    -0.12         &                  &    -0.23         \\
                &                  &   (0.25)         &                  &   (0.16)         &                  &   (0.19)         &                  &   (0.22)         \\
\midrule
Observations    &     4120         &     4120         &      718         &      718         &     4137         &     4137         &      675         &      675         \\
\bottomrule
\multicolumn{9}{l}{\footnotesize Standard errors in parentheses}\\
\multicolumn{9}{l}{\footnotesize \sym{*} \(p<0.10\), \sym{**} \(p<0.05\), \sym{***} \(p<0.01\)}\\
\end{tabular}
}
{
\def\sym#1{\ifmmode^{#1}\else\(^{#1}\)\fi}
\begin{tabular}{l*{8}{c}}
\toprule
                &\multicolumn{1}{c}{}&\multicolumn{1}{c}{model\_14\_b}&\multicolumn{1}{c}{model\_14\_c}&\multicolumn{1}{c}{model\_14\_d}&\multicolumn{1}{c}{model\_14\_e}&\multicolumn{1}{c}{model\_14\_f}&\multicolumn{1}{c}{model\_14\_g}&\multicolumn{1}{c}{model\_14\_h}\\
\midrule
L.IMPUTED\_CRRA\_1&     0.49         &                  &    -0.33         &                  &    -0.05         &                  &     1.66         &                  \\
                &   (0.66)         &                  &   (1.20)         &                  &   (0.62)         &                  &   (1.87)         &                  \\
\addlinespace
L.IMPUTED\_CRRA\_4&                  &    -5.05         &                  &    -1.21         &                  &    -2.66         &                  &     0.31         \\
                &                  &   (7.97)         &                  &   (4.79)         &                  &   (4.82)         &                  &   (5.78)         \\
\midrule
Observations    &     3508         &     3508         &      616         &      616         &     3531         &     3531         &      571         &      571         \\
\bottomrule
\multicolumn{9}{l}{\footnotesize Standard errors in parentheses}\\
\multicolumn{9}{l}{\footnotesize \sym{*} \(p<0.10\), \sym{**} \(p<0.05\), \sym{***} \(p<0.01\)}\\
\end{tabular}
}
{
\def\sym#1{\ifmmode^{#1}\else\(^{#1}\)\fi}
\begin{tabular}{l*{8}{c}}
\toprule
                &\multicolumn{1}{c}{}&\multicolumn{1}{c}{model\_15\_b}&\multicolumn{1}{c}{model\_15\_c}&\multicolumn{1}{c}{model\_15\_d}&\multicolumn{1}{c}{model\_15\_e}&\multicolumn{1}{c}{model\_15\_f}&\multicolumn{1}{c}{model\_15\_g}&\multicolumn{1}{c}{model\_15\_h}\\
\midrule
L.IMPUTED\_CRRA\_1&    -1.16         &                  &    -0.59         &                  &    -1.28         &                  &    -0.45         &                  \\
                &   (0.93)         &                  &   (1.41)         &                  &   (0.86)         &                  &   (1.63)         &                  \\
\addlinespace
L.IMPUTED\_CRRA\_4&                  &    -2.01         &                  &    -3.27         &                  &     4.66         &                  &    -4.79         \\
                &                  &  (12.42)         &                  &   (5.21)         &                  &  (12.04)         &                  &   (5.99)         \\
\midrule
Observations    &     3136         &     3136         &      539         &      539         &     3150         &     3150         &      504         &      504         \\
\bottomrule
\multicolumn{9}{l}{\footnotesize Standard errors in parentheses}\\
\multicolumn{9}{l}{\footnotesize \sym{*} \(p<0.10\), \sym{**} \(p<0.05\), \sym{***} \(p<0.01\)}\\
\end{tabular}
}

		\label{table:result-entrepreneur}
		\begin{tablenotes}[flushleft] \footnotesize
			\item \textit{Note:} The table present 2-step system GMM estimates of $\delta$ from \eqref{eq:empirical-specification} by subsamples on criteria of business ownership (NLSY79), business ownership (created variable), self-proclamation entrepreneur. All specifications include a full set of controls and fixed effect as described in \Cref{table:main-result}. For each panel, the first two lines present point estimates and standard error of $\gamma$ under the assumption of time-invariant risk preference and the next two lines time-variant risk preference. Standard errors are clustered by mother and shown in parentheses. 
			\item Significant levels: \sym{*} \(p<0.10\), \sym{**} \(p<0.05\), \sym{***} \(p<0.01\).
		\end{tablenotes}
	\end{threeparttable}
\end{table}


Before attempting to interpret our results, we now look at the effects of risk aversion by various indicators of business ownership and entrepreneurship. \Cref{table:result-entrepreneur} reports GMM estimates by subgroups of employees and entrepreneurs. We also separate \Cref{table:result-entrepreneur} into 5 panels of parental inputs. For each panel, the first two columns report estimates when sample is subdivided by whether mothers have ever owned a business measured by an indicator taken from the NLSY79. GMM estimates with sampling based on the constructed indicator of business ownership are shown in column 3 and 4. The last two columns present estimates by self-proclamation of being an entrepreneur. For each panel, the first two lines present point estimates and standard error of $\gamma$ under the assumption of time-invariant risk preference and the next two lines time-variant risk preference. With these classifications, no discernible effects are found for samples segregated by the measure from the NLSY79 dataset. Again, statistically significant and negative effects are found for goods index but only for business owners, measured by actual number of business owned, and those who identify as entrepreneurs. A unit increase in CRRA coefficient induces a 0.07 (about 0.08 SD) drop in goods investment for the group of mothers who are business owners under the constant risk preference assumption. A unit increase in RRA coefficient, ceteris paribus, brings investment down by 0.48 point (about 0.54 SD) for the sample of mothers who identify as entrepreneurs under the assumption of floating risk preference. Additionally, under the same assumption, we observe a drop of 0.35 point (about 0.4 SD) in time investment in response to a unit increase in RRA coefficient for those who do not own any business.  

One pattern persists throughout these different specifications and subsamplings, namely the statistically significant and sizeable negative effect of risk aversion on goods investment under the assumption of floating risk preference. The negative effect is driven mostly by the group of mothers with entrepreneur identity whose children are either between 3-5 (preschooler) or 10-14 (middle-schooler). Despite the persistent trend, it is hard to interpret this finding or investigate the underlying mechanism due to the lack of a suitable theory linking parental risk preference and investment in children. The most we can offer is a conjecture. 

Although entrepreneurs are generally more willing to take risks, a comparison among entrepreneurs reveal that those who are more risk averse have firms of smaller size and lower leverage \citep{herranz2015entrepreneurs}. If the same rationale applies to investment in children, we would expect mothers who identify as entrepreneurs to shrink investment during periods where substantial ``risks" are perceived. Thus far, we have refrained from discussing what risks mean in the context of investment in children because such an explicit discussion warrants well-grounded theoretical and empirical underpinnings on the aspects which perceived risks manifest and influence investment decisions. Traditionally, risk is associated with uncertainties in ability, cognitive or non-cognitive outcomes, or market outcomes. Following the narrative of \citet{knudsen2006economic} and \citet{cunha2007technology} on ``sensitive period," defined as periods in which the formation of skills are more productive, we hypothesise that mothers perceive substantial risks when their children enter these ``sensitive" age-range when considerable changes typically occur. It is an indisputable fact that children experience drastic changes in temperament once they enter puberty. Accordingly, it is possible that their erratic behaviours constitute a risk factor from the perspective of mothers. The situation aggravates when mothers have little information on their children. Despite the substantial risks mothers face when their children enter puberty, they did have information on the children's previous behaviours. Mothers whose children are about to enter preschool, on the other hand, have very limited information on how their children would behave when interacting with other children or unfamiliar adults. In response to these perceived risks, mothers with entrepreneur identity reduce investment in goods input which include books, magazines, and CD player for children between 3-5, and books, newspaper and musical instrument for children between 10-14. It is possible that maternal risk preference affects goods investment but not other dimensions of parental inputs because the similarity in pecuniary nature between investment in goods and investment in physical capital or infrastructure for a firm. Again, this hypothesis hinges crucially on the premise that self-identified entrepreneur mothers react to risks in investment in children the same way they would in an entrepreneurial setting. How plausible this precondition is will be left for future investigation.    



%===================================================================================================================================================================
%===================================================================================================================================================================

\section{Robustness checks}
In this section, we perform various checks to ensure the robustness of our model. These include re-estimating the model with the use of an ordinal variable indicating risk aversion instead of a cardinal measure ($\gamma$), and incorporating information on cognitive and non-cognitive ability and behavioural issues of the child. 

\subsection{Risk as category}
Now we focus on the treatment of the variable that acts as a surrogate for risk preference, i.e., how we incorporate responses to the sequence of risk preference in our model. As has been done thus far, one way is to impute and turn these responses into a single cardinal variable which aids in interpretation. In so doing, however, we impose several identifying restrictions crucial to the imputation process as noted in Section 3. Another approach which relaxes these restrictions is to include an indicator for each response group of respondents. Additionally, to separate the old series and new series, we interact the categorical variable with an indicator variable which takes value 1 for the new series and 0 otherwise. \Cref{table:robust-1} repeats the estimation done in \Cref{table:main-result} when the cardinal variable $\gamma$ representing RRA coefficient is replaced by a categorical variable indicating response group with group 4, the least risk averse, acting as the baseline. 

Since there is no distinguishing feature of risk preference when the ordinal variable is employed for the OLS specification, the general effect of risk aversion for each group is reported. It should be kept in mind that these are obtained without any assumptions on the temporal properties of risk preference. Results of this exercise are available in Appendix B. Overall, OLS estimates are only significant for emotional support percentile score in the subsample of mothers who are most risk averse based on the old measure. On the other hand, in the GMM specifications, risk  aversion is found to have statistically significant and large negative effects on general investment (HOME index), goods investment and cognitive stimulation. Unfortunately, however, it is difficult to determine whether these estimates carry any meaningful interpretation as the categorisation is static, meaning it does not account for the fact that participants of the lifetime income gamble might change their responses in subsequent rounds.



\subsection{Controlling for child's test scores}

Thus far, we have not include any measures indicative of children's ability since we are not trying to explicitly estimate an empirical model of child development in which mothers update investment, subject to their degree of risk aversion, in response to information on cognitive and non-cognitive ability as well as behaviours problem of the child. Such a study, ideally should follow closely the work of \citet{cunha2007technology} and \citet{cunha2010estimating} with appropriate modification to accommodate identification of parental risk preference. That said, to see how the estimates would change in response to information on children's ability, we control for a battery of measures of cognitive and non-cognitive ability, and behavioural issues measured by three Peabody Individual Achievement Test (PIAT) scores including math, reading cognition and comprehension; Peabody Picture Vocabulary Test (PPVT); and Behavioural Problem Index (BPI). These tests are administered at different starting age of children with the earliest being PPVT for children age 3 and above. Information on BPI is collected for children age 4 and above while PIAT are only administered to children whose age are 5 years and above. 

We progressively include the measure by age requirement, i.e., PPVT in the first specification, PPVT and BPI in the second specification, and all measures, PIAT inclusive in the last specification. All measures are treated as endogenous to parental inputs. Due to the age requirement of these tests, later specifications naturally have fewer observations. We focus on the effects of maternal risk aversion on parental inputs for children age 3 and above.





%===================================================================================================================================================================
%===================================================================================================================================================================


\section{Conclusion}
Even though mothers face various types of risks when making investment in their children, theoretical and empirical studies linking maternal, or more broadly parental, risk aversion and investment in children is severely lacking. By linking mothers and children in the NLSY79, we are able to examine how maternal risk aversion affects various parental inputs. We measure risk preference of mothers via imputations taking as inputs survey responses to lifetime income gamble. Our imputations follow \citet{kimball2008imputing,kimball2009risk} to separate potential confounding factors of mother's characteristics from risk preference and generate a cardinal measure of RRA. We examine the effects of risk aversion on parental inputs under two assumptions, i.e., time-invariant CRRA and time-variant RRA. It should be stressed that we do not attempt to validate either assumption in the paper but rather compare how the effects vary under these two assumptions. To deal with autocorrelation among investments in different periods, we adopt the 2-step system GMM by \citet{arellano1991some} and \citet{blundell1998initial} with orthogonal deviations \citep{arellano1995another} and corrected errors \citep{windmeijer2005finite}. Our results indicate a statistically significant and economically meaningful effect of time-varying risk aversion on goods investment for children entering preschool (3-5) or middle-school (10-14). On average, children within these age-range of mothers who identify as entrepreneurs receive up to 0.6 point (0.7 SD) lower in terms of goods investment if their mothers are 1.15 SD more risk-averse. 

We hypothesise that this pattern arise because mothers act as if investment in goods are, in nature, similar to investment in an entrepreneurial setting. A reasonable explanation for this assumption is the resemblance between goods investment for children and physical capital investment for business, both of which incur direct pecuniary costs unlike other dimensions of parental inputs. Consequently, more risk-averse mothers reduce their investment in goods for children when they enter ``sensitive" periods, i.e., when they enter the 3-5 and 10-14 age range. These periods are potentially perceived more risky to mothers who accordingly reduce investment in goods in an attempt to minimise exposure to risk. At this stage, this line of explanation is purely conjectural and warrants further investigation. Future study might attempt to explicate the relationship between parental risk aversion, that is including risk preference of both parents, and the process of child development. In addition, it would also be immensely beneficial to lay the groundwork for future study on investment in children by a detailed characterisation of investment in children. More elaborately, said study would ideally identify sources of disturbances to investment in children and how these disturbances influence investment at different stage of a child's development.




\pagebreak

\bibliographystyle{apalike}
\bibliography{bibliography.bib}


\pagebreak

\section*{Appendix}
\appendix

\setcounter{secnumdepth}{3}
\counterwithin{equation}{section}
\counterwithin{table}{section}

\section{First-stage MLE of risk preference}
\subsection{Time-invariant risk preference}
Under the assumption of time-invariant risk preference, our estimates are most conservative in that the probability distribution functions, and hence the derived conditional probability, depend only on survey responses to the question on lifetime income gamble. The probability of being in category $j$ for an individual who answered in only one wave is 
\parskip 0cm
\begin{equation}
P(c=j)=P(\log\underline{\theta}_j<x<\log\bar{\theta}_j)=\Phi\Big({\log(\bar{\theta}_j)-\mu-b\over\sqrt{\sigma_x^2+\sigma_{eq}^2+\sigma_{\kappa q}^2}}\Big)-\Phi\Big({\log(\underline{\theta}_j)-\mu-b\over\sqrt{\sigma_x^2+\sigma_{eq}^2+\sigma_{\kappa q}^2}}\Big) 
\label{eq:A-likelihood-constant-univariate}	
\end{equation}
$\mu$ and $\sigma_x$ are respectively the mean and standard deviation of true log risk tolerance. $\sigma_{eq}^2$ is variance in an individual's transitory response error for a particular wave and question type $q$ while $\sigma_{\kappa q}^2$ is variance in an individual's persistent response error for question type $q$. $\bar \theta_j$ and $\underline{\theta}_j$ are upper and lower bounds on the coefficient of risk tolerance. $b$ denotes status quo bias (common bias across individuals to questions of type $q$) and $\sigma_e$ standard deviation of transitory response error, assumed to be purely random and independent of an individual's true risk tolerance or any other attributes. The coefficient of risk aversion, $\gamma$, is then imputed using the conditional expectation that an individual belongs to a certain risk category $j$,

\begin{equation}
\mathbb E(\gamma|c=j)=\exp\Big(-\mu+{\sigma_x^2\over 2}\Big){\Phi\Big({\log(\bar{\theta}_j)-\mu-b+\sigma_x^2\over \sigma_{\xi}}\Big)-\Phi\Big({\log(\underline{\theta}_j)-\mu-b+\sigma_x^2\over \sigma_{\xi}}\Big)\over \Phi\Big({\log(\bar{\theta}_j)-\mu-b\over \sigma_\xi}\Big)-\Phi\Big({\log(\underline{\theta}_j)-\mu-b\over \sigma_\xi}\Big)}
\label{eq:A-conditional-expectation}
\end{equation}

If she participated in exactly two waves, the probability of her being in category $j$ in wave $w$ and category $k$ in wave $w'$ is
\begin{equation}
P(c_w=j, c_{w'}=k)=\vec{\Phi}(\bar N_{jq}, \bar N_{kq'},\rho)+\vec{\Phi}(\underline N_{jq}, \underline N_{kq'},\rho)-\vec{\Phi}(\underline N_{jq}, \bar N_{kq'},\rho)-\vec{\Phi}(\underline N_{jq}, \bar N_{kq'},\rho)
\label{eq:A-likelihood-constant-bivariate}
\end{equation}
where $\vec{\Phi}$ is the bivariate normal cumulative distribution function, $\bar N_{jq}=(\log \bar \theta_j-\mu-b_q)/\sigma_q,\bar N_{kq'}=(\log \bar \theta_k-\mu-b_{q'})/\sigma_{q'}, \underline N_{jq}=(\log \underline \theta_j-\mu-b_q)/\sigma_q, \underline N_{kq'}=(\log \underline \theta_k-\mu-b_{q'})/\sigma_{q'}$, and $\rho$ is the correlation between answers across the two waves. If respondents participated in more than two waves, the form of \eqref{eq:A-likelihood-constant-bivariate} alters accordingly to accommodate multivariate normal distribution.

\subsection{Time-varying risk preference}
Under the assumption of time-variant risk preference, we augment our estimation with both time-invariant and time-variant characteristics of participants. The former comprise gender, race, parents' years of schooling, an indicator of whether the respondent lived with both biological parents at age 14, AFQT score (2006 revised edition). The latter consist of marital status, age, educational level,\footnote{Educational level is allowed to vary with time to capture the potential endogeneity of education, i.e., more educated individuals are less risk averse but less risk averse individuals may also choose to obtain more education.} log of income, and the number of weeks worked in the previous year by respondent and respondent's spouse. To further control for fluctuations in macroeconomic condition at the time the questions on risk preference were asked, we also include the unemployment rate and index of consumer sentiment (ICS) as covariates. Under this assumption, the probability of being in category $j$ for an individual $i$ who answered in only one wave is
\begin{equation}
	P(c=j)=P(\log\underline{\theta}_j<x<\log\bar{\theta}_j)=\Phi\Big({\log(\bar{\theta}_j)-\mu-b-\beta X_{it}\over\sqrt{\sigma_x^2+\sigma_{eq}^2+\sigma_{\kappa q}^2}}\Big)-\Phi\Big({\log(\underline{\theta}_j)-\mu-b-\beta X_{it}\over\sqrt{\sigma_x^2+\sigma_{eq}^2+\sigma_{\kappa q}^2}}\Big) 
	\label{eq:A-likelihood-floating-univariate}	
\end{equation}
where $X_{it}$ denotes the set of characteristics of the respondents at the time of the survey, $t$. We can modify \eqref{eq:A-likelihood-constant-bivariate} in the same spirit to formulate the likelihood of respondents who participated in more than one wave. 

\subsection{MLE results}

\Cref{table:appendix-mle} reports MLE results for 2 specifications, time-invariant with stringent constraints that risk preference only depend on survey responses, and time-variant which incorporates information on several characteristics of participants. Since we only have one response from any individuals for the newer series, the identification problem of risk preference, especially with regards to the estimation of persistent error of the SQB-free questions, is under-identified. Nevertheless, other parameters are very precisely estimated.
\parskip 0cm
\begin{table}[!htbp]
	\centering \setlength{\extrarowheight}{0.3em}
	\caption{Distribution of log risk tolerance: Maximum likelihood estimates}
	\begin{threeparttable}
		{
\begin{tabular}{l *{2}{c}}
	\toprule
	\textbf{Parameter} & \thead{Time-invariant\\ risk preference} & \thead{Time-variant\\ risk preference} \\
	\midrule
	\multicolumn{3}{l}{Log risk tolerance}\\
	\hspace{2em} Mean $\mu$ & -0.049 & -0.116 \\
	 & (0.035) & (0.191) \\
	\hspace{2em} Standard deviation $\sigma_x$ & 0.894 & 0.873 \\
	 & (0.036) & (0.023) \\
	Status quo bias $b_o$ & 0.105 & -0.058 \\
	& (0.040) & (0.201) \\
	
	\addlinespace 
	\multicolumn{3}{l}{Response error standard deviation}\\
	\hspace{2em} Original question, & 1.486 & 1.461 \\
	\hspace{3em} transitory $\sigma_{eo}$ & (0.016) & (0.016) \\
	\hspace{2em} Original question, & 0.348 & 0.348 \\
	\hspace{3em} persistent $\sigma_{ko}$ & (0.071) & (0.067) \\
	\hspace{2em} SQB-free question,  & 2.176 & 2.114 \\
	\hspace{3em} transitory $\sigma_{ef}$ & (0.229) & (0.157) \\
	\hspace{2em} SQB-free question & 0.520 & 0.376 \\
	\hspace{3em} persistent $\sigma_{kf}$ & (0.945) & (0.859) \\
	\midrule
	Number of individuals & 9279 & 9279 \\
	Number of responses & 38999 & 38999 \\
	Log-likelihood & - 37822.7 & -37507.9 \\
	\bottomrule
\end{tabular}
}
		\begin{tablenotes}[flushleft]\footnotesize
			\item The table reports MLE of parameters of log risk tolerance and response error under the assumption of time-invariant risk preference following \eqref{eq:A-likelihood-constant-univariate} and \eqref{eq:A-likelihood-constant-bivariate}, and time-invariant risk preference following \eqref{eq:A-likelihood-floating-univariate}. Standard error in parentheses.
		\end{tablenotes}
	\end{threeparttable}
	\label{table:appendix-mle}
\end{table}


\parskip 0cm

\pagebreak



\section{Robustness checks}
\subsection{Estimation with categorical variable}

\begin{ThreePartTable}
	\begin{TableNotes}[flushleft]\footnotesize
		\item \footnotesize The table reports estimates for $\delta$ in \eqref{eq:empirical-specification} with the categorical variable indicating response group instead of $\gamma$. Group 4, least risk aversion, is treated as baseline. For the OLS specification, the estimates for respondents in group 1 of the new series are excluded due to collinearity. Other details are similar to \Cref{table:main-result}. Standard errors in parentheses.
		\item \sym{*} \(p<0.10\), \sym{**} \(p<0.05\), \sym{***} \(p<0.01\).\\
	\end{TableNotes}
	{
\def\sym#1{\ifmmode^{#1}\else\(^{#1}\)\fi}
\setlength{\extrarowheight}{0em}
\begin{longtable}{l*{3}{c}}
	\caption{The effect of risk preference on Investment in children \\with response category for risk} \\
\toprule\endfirsthead\midrule\endhead\endfoot\endlastfoot
                &\multirow{2}{*}{\thead{OLS \\with fixed effect}}&\multicolumn{2}{c}{\textbf{2-step GMM}}\\ \cline{3-4}
                & & Time-invariant & Time-variant\\
\midrule
\addlinespace
\multicolumn{4}{l}{\textit{Panel A: HOME index}}\\
Group 4 $\times$ NEW&    0.077         &   -0.465         &   -1.547         \\
                &  (0.191)         &  (0.304)         &  (1.071)         \\

Group 3 &   -0.032         &   -0.009         &   -0.138         \\
                &  (0.063)         &  (0.042)         &  (0.137)         \\

Group 3 $\times$ NEW&   -0.212         &   -0.684\sym{**} &    0.139         \\
                &  (0.238)         &  (0.336)         &  (2.479)         \\

Group 2 &   -0.070         &   -0.066         &   -0.007         \\
                &  (0.066)         &  (0.046)         &  (0.142)         \\

Group 2 $\times$ NEW&    0.098         &   -0.511\sym{*}  &   -2.095         \\
                &  (0.139)         &  (0.292)         &  (1.427)         \\

Group 1 &   -0.045         &   -0.038         &    0.065         \\
                &  (0.062)         &  (0.038)         &  (0.123)         \\

Group 1 $\times$ NEW&    -         &   -0.576\sym{*}  &   -1.371\sym{*}  \\
                &      -         &  (0.306)         &  (0.787)         \\
\midrule
Observations    &     4983         &     4983         &     4983         \\


\midrule
\addlinespace
\multicolumn{4}{l}{\textit{Panel B: Goods index}}\\
Group 4 $\times$ NEW&   -0.226         &   -0.619\sym{**} &   -3.117\sym{*}  \\
                &  (0.248)         &  (0.275)         &  (1.651)         \\

Group 3 &   -0.090         &   -0.036         &   -0.125         \\
                &  (0.064)         &  (0.046)         &  (0.097)         \\

Group 3 $\times$ NEW&   -0.168         &   -0.799\sym{***}&   -1.614         \\
                &  (0.286)         &  (0.307)         &  (2.656)         \\

Group 2 &   -0.064         &   -0.058         &   -0.095         \\
                &  (0.061)         &  (0.039)         &  (0.084)         \\

Group 2 $\times$ NEW&    0.024         &   -0.624\sym{**} &    2.304\sym{*}  \\
                &  (0.177)         &  (0.273)         &  (1.224)         \\

Group 1 &   -0.073         &   -0.029         &   -0.100         \\
                &  (0.055)         &  (0.038)         &  (0.098)         \\

Group 1 $\times$ NEW&    -         &   -0.688\sym{**} &   -2.229\sym{***}\\
                &      -         &  (0.272)         &  (0.721)         \\
\midrule
Observations    &     4968         &     4968         &     4968         \\
\midrule

\pagebreak 


&\multirow{2}{*}{\thead{OLS \\with fixed effect}}&\multicolumn{2}{c}{\textbf{2-step GMM}}\\ \cline{3-4}
& & Time-invariant & Time-variant\\
\midrule
\addlinespace
\multicolumn{4}{l}{\textit{Panel C: Time index}}\\
Group 4 $\times$ NEW&    0.184         &   -0.384         &   -0.743         \\
                &  (0.190)         &  (0.333)         &  (0.948)         \\

Group 3 &   -0.008         &    0.028         &   -0.087         \\
                &  (0.071)         &  (0.049)         &  (0.156)         \\

Group 3 $\times$ NEW&   -0.227         &   -0.460         &    0.677         \\
                &  (0.209)         &  (0.386)         &  (2.364)         \\

Group 2 &   -0.052         &   -0.023         &    0.031         \\
                &  (0.073)         &  (0.052)         &  (0.155)         \\

Group 2 $\times$ NEW&    0.135         &   -0.364         &   -2.302         \\
                &  (0.140)         &  (0.326)         &  (1.517)         \\

Group 1 &   -0.023         &   -0.008         &    0.135         \\
                &  (0.069)         &  (0.044)         &  (0.131)         \\

Group 1 $\times$ NEW&    -         &   -0.396         &   -0.793         \\
                &      -        &  (0.347)         &  (0.881)         \\
\midrule
Observations    &     4968         &     4968         &     4968         \\
\midrule


\addlinespace
\multicolumn{4}{l}{\textit{Panel D: Cognitive Stimulation score (\%)}}\\
Group 4 $\times$ NEW&   -6.113         &  -19.296\sym{*}  &  -62.202\sym{**} \\
                &  (6.772)         & (10.106)         & (30.760)         \\

Group 3 &    3.191         &    1.295         &   -0.934         \\
                &  (2.212)         &  (1.437)         &  (5.007)         \\

Group 3 $\times$ NEW&   -3.851         &  -16.324         &   40.546         \\
                &  (5.375)         & (10.597)         & (88.072)         \\

Group 2 &    1.860         &    0.378         &    0.118         \\
                &  (2.321)         &  (1.419)         &  (5.744)         \\

Group 2 $\times$ NEW&    2.253         &  -12.288         &  -27.806         \\
                &  (4.976)         & (10.072)         & (30.805)         \\

Group 1 &    1.625         &    0.180         &    1.916         \\
                &  (2.142)         &  (1.247)         &  (4.676)         \\

Group 1 $\times$ NEW&   -         &  -13.350         &  -60.627\sym{**} \\
                &      -         & (10.388)         & (27.362)         \\
\midrule
Observations    &     4224         &     4224         &     4224         \\
\midrule
\multicolumn{4}{r}{\textit{Continue on next page}}\\

\pagebreak
&\multirow{2}{*}{\thead{OLS \\with fixed effect}}&\multicolumn{2}{c}{\textbf{2-step GMM}}\\ \cline{3-4}
& & Time-invariant & Time-variant\\
\midrule
\addlinespace
\multicolumn{4}{l}{\textit{Panel E: Emotional Support score (\%)}}\\
Group 4 $\times$ NEW&    7.308         &    2.507         &  -52.262         \\
                &  (9.331)         & (12.533)         & (42.043)         \\

Group 3 &    0.949         &   -0.414         &    1.055         \\
                &  (3.871)         &  (1.890)         &  (6.094)         \\

Group 3 $\times$ NEW&    2.464         &   -1.711         &  186.316         \\
                &  (9.987)         & (14.581)         &(164.529)         \\

Group 2 &   -3.420         &    1.828         &    2.432         \\
                &  (3.413)         &  (1.526)         &  (4.526)         \\

Group 2 $\times$ NEW&    6.591         &    0.121         &   15.130         \\
                &  (6.449)         & (12.383)         & (41.789)         \\

Group 1 &   -6.468\sym{**} &   -0.991         &   -3.042         \\
                &  (3.058)         &  (1.507)         &  (5.397)         \\

Group 1 $\times$ NEW&    -         &    1.306         &  -39.663         \\
                &      -         & (12.670)         & (36.488)         \\
\midrule
Observations    &     3783         &     3783         &     3783         \\
\bottomrule

\insertTableNotes
\end{longtable}
}

\end{ThreePartTable}



\end{document}
