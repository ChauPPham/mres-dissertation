\documentclass[]{article}

%opening
\title{Parental risk aversion and investment in children}
\author{Chau Pham}
%\date{} for no date shown
\usepackage{amsmath, amsfonts, amssymb, amsthm, mathtools, mathrsfs}
\usepackage[margin = 1.0in]{geometry}
\usepackage{bbold, soul}
\usepackage{tgpagella}
\usepackage{braket, setspace, parskip, enumitem}
\usepackage[colorlinks, citecolor = blue]{hyperref}
\usepackage[nameinlink, noabbrev]{cleveref}	% Use \Cref always 
\usepackage{caption}
\usepackage{threeparttable, makecell, cellspace, multirow, booktabs, diagbox, tabularx, longtable, threeparttablex}
\usepackage{verbatim}
\usepackage{natbib}
\usepackage[bottom]{footmisc}	% keep footnotes at end of the page
\usepackage{xcolor}

\newcolumntype{Y}{>{\centering\arraybackslash}X}

\renewcommand\theadalign{bc}
\renewcommand\theadfont{\bfseries}
\renewcommand\theadgape{\Gape[4pt]}
\renewcommand\cellgape{\Gape[4pt]}

%---- Simple table template
\begin{comment} 
	\begin{table}[!htbp]\centering
		\begin{threeparttable}
			\setlength{\extrarowheight}{0.2em}
			\begin{tabular}
				content...
			\end{tabular}
			%\caption{text}
			%\label{key}
		\end{threeparttable}
	\end{table}
\end{comment}

%\setlist[enumerate]{label=(\roman*)}

%\setlength{\parindent}{0pt}

%\captionsetup[figure]{labelfont=bf}
\captionsetup[table]{labelfont=bf}

\begin{document}

\maketitle
\onehalfspacing

\section{Introduction}
Investment in human capital has been considered risky since the works of \citet{becker2009human, levhari1974effect} and \citet{schultz1971investment}. Returns to investment, ability, probability of getting a good job are few of the various accounts of risk. A traditional approach on the topic examines how an individual's risk preference affects her investment decision. This line of inquiry is, however, valid insofar as the decision maker is the individual under study. College choices, tertiary education, specific skills training largely fall under the domain of individual choice wherein an individual is the decision maker and she exercises scrutiny before making her choice. For children and teenagers, this is often not the case as they lack the appropriate capacity to make an informed choice while various factors such as pecuniary issues are beyond their control. At the same  time, a vast literature on early childhood intervention has deemed such investment substantially beneficial for future outcomes of children.\footnote{See \citet{Currie2001} and \citet{NORES2010} for a comprehensive review on a large number of childhood intervention programs.} Furthermore, as highlighted by {\color{red}FIND CITATION FOR THIS Banerjee?...}, early interventions are most effective in levelling the playground for disadvantaged children. Thus, parents' investment during early childhood plays a pivotal role in shaping the children's future paths. If investment in human capital is risky, the importance of understanding the effect of parental risk preference on children investment naturally follows.

In general, the literature on parental risk aversion is scarce whilst results are inconclusive. In direct relation to our study, \citet{sovero2018risk} uses height-for-age, and BMI-for-age along with other several spending categories to measure investment of Mexican parents. Risk attitudes of parents are elicited using a series of questions with multiple price list structure. Her results indicate a positive linkage between risk averse mothers and their investment on sons, i.e., risk averse mothers invest more on sons than daughters. Using data on Uganda households, \citet{tabetando2019parental} confirms the positive relation between risk aversion and investment, measured by level of educational expenditure and its share of household budget. Although wealthier households tend to be more risk averse, the relationship is reversed for poor households. That is, risk aversion depresses investment in credit constrained households.      

Other works involving risk preference of parents focus on educational outcomes. With the exception of \citet{leonardi2007parents} which found no effect of risk aversion on secondary school track choice for the Italian sample, the majority reported negative effects of risk aversion on educational outcomes including test scores in the US \citep{brown2012parental}, college enrollment in Italy \citep{checchi2014parents}, cognitive outcomes in Indonesia \citep{hartarto2023parental}. Furthermore, \citet{frempong2021risk} found a positive association between risk aversion and child labour while there is evidence that increased child labour impedes learning \citep{HEADY2003385, bezerra2009impact}.  



\section{Data \& methodology}





\section{Results}





\section{Robustness checks}





\section{Conclusion}

\pagebreak

\bibliographystyle{apalike}
\bibliography{bibliography.bib}

\end{document}
